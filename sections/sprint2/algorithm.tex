For determining whether a route is a good fit, the system uses an algotihm.
This algorithm is based on a simplified version of how \citet{ghoseiri2011real} determines a match.
In the paper they take many aspects into account when determining matches, such as smoking, gender and age preferences.
These are all valid but the aspects that this algorithm focuses on are time and distance.

\subsubsection{The main algorithm}
Lets say we want to determine wheter or not person $A$ could pickup person $B$ on his way through route $r$ when person $B$ wants to go from his start position $s$ to his goal position $g$.
If $A$ could pickup $B$ then we want to know how good a match it is.

This result could be expressed as a value from 0 to 1 which tells how close the match is where 0 is not a match and 1 is when it is litteraly \emph{on} the way there.

To summerize we have an algortihm which takes in a route $r$ and two positions $s$ and $g$, and returns a value between 0 and 1.

In the algorthm there are two aspects which needs to be evaluated: Time and distance.
If we imagine that these have already been solved with a similar interface, where they take a route and two points as input and return a value from 0 to 1 depending on how close they match, but only considering one of the aspects.

If one of the aspects is deemed unacceptable by giving it a score of 0 then the whole match is unacceptable and is given the same final score.
From the scores the average score could be expressed as $\frac{d+t}{2}$ where $d$ is the distance score and $t$ is the time score.
But to increase flexibility of the algorithm two constants are introduced $d_\gamma$ and $t_\gamma$.
These constant enables us to weight the time and distance score independant of each other by expanding the expression to $\frac{d\times d_\gamma+t\times t_\gamma}{d_\gamma+t_\gamma}$.

\begin{algorithm}
	\caption{The Time Distance Analyser pseudocode}
	\label{alg:timedistanalyser}
	\begin{algorithmic}[1]
		\Require 
		\Statex $d_\gamma\in \mathbb{R}_{>0}$ : the modifier for distance
		\Statex $t_\gamma\in\mathbb{R}_{>0}$ : the modifier for time
		\Statex 
		\Function{TimeDistanceAnalyser}{$r, s, g$}
			\State $d\gets$\Call{DistanceAnalyser}{$r, s, g$}
			\State $t\gets$\Call{TimeAnalyser}{$r, s, g$}
			
			\If{$d > 0 \wedge t > 0$}
				\State \Return $\frac{d\times d_\gamma+t\times t_\gamma}{d_\gamma+t_\gamma}$
			\Else
				\State\Return 0
			\EndIf
		\EndFunction
	\end{algorithmic}
\end{algorithm}

The final algorithm can be seen in Algorithm \ref{alg:timedistanalyser}.
Line 2 and 3 is the calls to the algorithms that solve the sub problems and line 4 to 8 is the logic that was explained in previous paragraph.
The complexity of the algortihm is dependant on the two calls on line 2 and 3, because the rest of the algorim is input independant and therefore $O(1)$ if we ignore line 2 and 3.

\subsubsection{Distance analyser}

Lets take a look at the sub problems.

First we have the distance analyser, this algorithm takes a route and two end points and figures out the score for the detour when only considering the distance.
To do this we need to define how long the longest acceptable detour should be.
We also need a way of figuring out the distance between two disconnected locations. 
For this the function for euclidian distances would work since it only gets significantly inaccurate on larger scales.
This means that we have the following function.

\begin{align*}
	dist : \mathbb{L}\times\mathbb{L} &\rightarrow \mathbb{R}_{\geq 0}\\
	a, b &\mapsto \sqrt{(a_x - b_x)^2 + (a_y - b_y)^2}
\end{align*}

In the function the symbol $\mathbb{L}$ is used to denote the set of all possible locations.


\begin{algorithm}
	\caption{The Distance Analyser pseudocode}
	\label{alg:distanalyser}
	\begin{algorithmic}[1]
		\Require 
		\Statex $\beta \in \mathbb{R}_{>0}$ : the largest acceptable detour length
		\Statex 
		\Function{DistanceAnalyser}{$r, s, g$}
			\State $r_s \in r$ : the closest point to $s$
			\State $r_g \in r$ : the closest point to $g$
			\State $d_s\gets 2\times dist(r_s, s)$\Comment Pickup detour distance
			\State $d_g\gets 2\times dist(r_g, g)$\Comment Set off detour distance
			\State\Return $1-\frac{d_s + d_g}{\beta}$
		\EndFunction
	\end{algorithmic}
\end{algorithm}

The distance analyser algorithm, as seen in Algorithm \ref{alg:distanalyser}, finds the points on the route that are closest to the start and end points.
From those points the algorithm approximates the detour required to take by the driver to pick up the passenger.


\newpage

\todo{This is a draft}
\todo{Algorithm structure: Idea, "toy example", pseudocode: top-down}
$\mathbb{L}$ is the set of all possible locations.

\begin{algorithm}
	\caption{The Time Analyser pseudocode}
	\begin{algorithmic}[1]
		\Require 
		\Statex $\delta\in \mathbb{R}_{>0}$ : the acceptable time difference
		\Statex $\gamma\in\mathbb{R}_{>0}$ : the translation from distance to time
		\Statex $dist : l_1,l_2 \in \mathbb{L} \rightarrow \mathbb{R}_{\geq 0}$ : the distance between $l_1$ and $l_2$
		\Statex $time : l_1,l_2 \in \mathbb{L} \rightarrow \mathbb{R}_{\geq 0}$ : the time difference between $l_1$ and $l_2$
		\Statex 
		\Function{TimeAnalyser}{$r, g$}
			\State $r_g \in r$ is the closest point to $g$
			\State $t_\delta\gets time(r_g, g) - 2\times dist(r,g)\times\gamma$;
			\State\Return $\frac{t_\delta}{\delta}$
		\EndFunction
	\end{algorithmic}
\end{algorithm}

Analysing the function Time Analyser we can see that on line 2 it should iterate over all locations in $r$ to make sure that $r_g$ is the closest possible point.
The complexity of this operation should be $\Theta(|r|)$ where $|r|$ is the number of locations in $r$.\todo{fix thetas to O's}
Moving on to line 3 we have some basic arithmetic operations and the function call $time$.
A possible heuristic algorithm for time would be to find the distance between the two points and multiply it with the time it takes to travel one distance unit. 
This is an optimistic approach since it assumes that there is a path directly from $r_g$ to $g$.

In order to make this work for both stability and similarity

If this heuristic approach is used then the complexity of line 3 would be $\Theta(1)$.
Because line 4 is only using basic arithmetic operations then this line is also $\Theta(1)$ making the complexity of this function $\Theta(|r|)$.

The Distance Analyser uses a distance function.
This only need to be guiding so a heuristic function should be sufficient.
A possible implementation is to find the direct distance between the locations.
This would be a $\Theta(1)$ time function.
The function it self finds the closest points on route $r$ to the endpoints $g$ and $s$ in line 2 and 3.
This is done in $\Theta(|r|)$ time.
When determining 

Line 4 and 5 uses the distance function which we earlier said was $\Theta(1)$, and line 6 uses basic arithmetic operations so the complexity of that is also $\Theta(1)$.
Making this functions complexity $\Theta(|r|)$.

The Time Distance Analyser uses the Time Analyser and Distance Analyser function which were defined as being $\Theta(|r|)$.
And then does a check and some arithmetic operations.
The limiting factor is the called functions so the complexity of this function must still be $\Theta(|r|)$.

\begin{algorithm}
	\caption{The Analyse Route pseudocode}
	\begin{algorithmic}
		\Require
		\Statex $scores(r, r') \rightarrow \mathbb{Z}$ : a matrix of scores between two routes
		\Statex
		\Procedure{AnalyseRoute}{$r, R$}
			\State $r_s\in r$ : the start point of route $r$
			\State $r_g\in r$ : the end point of route $r$
			\ForAll{$r'\in R$}
				\State $r'_s\in r'$ : the start point of route $r'$
				\State $r'_g\in r'$ : the end point of route $r'$
				\State $score(r', r)\gets$\Call{TimeDistanceAnalyser}{$r', r_s, r_g$}
				\State $score(r, r')\gets$\Call{TimeDistanceAnalyser}{$r, r'_s, r'_g$}
			\EndFor
		\EndProcedure
	\end{algorithmic}
\end{algorithm}

The final procedure takes in a route and a set of routes and calls the Time Distance Analyser two times for each route in $R$.
If we consider $\langle R \rangle$ to be the average length of all routes in $R$ and $|R|$ to be the number of routes in $R$.
Then the complexity would be $\Theta(|R|\times(\langle R\rangle + |r|))$.
$|r|$ could be represented within the average.
So the reduced complexity is $\Theta(|R|\times\langle R\rangle)$.
As the number of routes get greater the average route length gets harder to change between calculations and approaches therefore a constant value.
This means that as the set gets larger we approach this complexity $\Theta(|R|)$.

