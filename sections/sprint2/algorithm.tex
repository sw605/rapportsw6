\todo{This is a draft}
\todo{Algorithm structure: Idea, "toy example", pseudocode: top-down}
$\mathbb{L}$ is the set of all possible locations.

\begin{algorithm}
	\caption{The Time Analyser pseudocode}
	\begin{algorithmic}[1]
		\Require 
		\Statex $\delta\in \mathbb{R}_{>0}$ : the acceptable time difference
		\Statex $\gamma\in\mathbb{R}_{>0}$ : the translation from distance to time
		\Statex $dist : l_1,l_2 \in \mathbb{L} \rightarrow \mathbb{R}_{\geq 0}$ : the distance between $l_1$ and $l_2$
		\Statex $time : l_1,l_2 \in \mathbb{L} \rightarrow \mathbb{R}_{\geq 0}$ : the time difference between $l_1$ and $l_2$
		\Statex 
		\Function{TimeAnalyser}{$r, g$}
			\State $r_g \in r$ is the closest point to $g$
			\State $t_\delta\gets time(r_g, g) - 2\times dist(r,g)\times\gamma$;
			\State\Return $\frac{t_\delta}{\delta}$
		\EndFunction
	\end{algorithmic}
\end{algorithm}

Analysing the function Time Analyser we can see that on line 2 it should iterate over all locations in $r$ to make sure that $r_g$ is the closest possible point.
The complexity of this operation should be $\Theta(|r|)$ where $|r|$ is the number of locations in $r$.\todo{fix thetas to O's}
Moving on to line 3 we have some basic arithmetic operations and the function call $time$.
A possible heuristic algorithm for time would be to find the distance between the two points and multiply it with the time it takes to travel one distance unit. 
This is an optimistic approach since it assumes that there is a path directly from $r_g$ to $g$.

In order to make this work for both stability and similarity

If this heuristic approach is used then the complexity of line 3 would be $\Theta(1)$.
Because line 4 is only using basic arithmetic operations then this line is also $\Theta(1)$ making the complexity of this function $\Theta(|r|)$.

 \begin{algorithm}
	\caption{The Distance Analyser pseudocode}
	\begin{algorithmic}[1]
		\Require 
		\Statex $\beta \in \mathbb{R}_{>0}$ : the acceptable detour length
		\Statex $dist : l_1,l_2 \in \mathbb{L} \rightarrow \mathbb{R}_{\geq 0}$ : the distance between $l_1$ and $l_2$
		\Statex 
		\Function{DistanceAnalyser}{$r, s, g$}
			\State $r_s \in r$ : the closest point to $s$
			\State $r_g \in r$ : the closest point to $g$
			\State $d_s\gets 2\times dist(r_s, s)$\Comment Pickup detour distance
			\State $d_g\gets 2\times dist(r_g, g)$\Comment Set off detour distance
			\State\Return $\frac{d_s + d_g}{\beta}$
		\EndFunction
	\end{algorithmic}
\end{algorithm}

The Distance Analyser uses a distance function.
This only need to be guiding so a heuristic function should be sufficient.
A possible implementation is to find the direct distance between the locations.
This would be a $\Theta(1)$ time function.
The function it self finds the closest points on route $r$ to the endpoints $g$ and $s$ in line 2 and 3.
This is done in $\Theta(|r|)$ time.
Line 4 and 5 uses the distance function which we earlier said was $\Theta(1)$, and line 6 uses basic arithmetic operations so the complexity of that is also $\Theta(1)$.
Making this functions complexity $\Theta(|r|)$.

\begin{algorithm}
	\caption{The Time Distance Analyser pseudocode}
	\begin{algorithmic}
		\Require 
		\Statex $d_\gamma\in \mathbb{R}_{>0}$ : the modifier for distance
		\Statex $t_\gamma\in\mathbb{R}_{>0}$ : the modifier for time
		\Statex 
		\Function{TimeDistanceAnalyser}{$r, s, g$}
			\State $d\gets$\Call{DistanceAnalyser}{$r, s, g$}
			\State $t\gets$\Call{TimeAnalyser}{$r, g$}
			
			\If{$d < 1 \vee t < 1$}
				\State $score\gets 1-\frac{d\times d_\gamma+t\times t_\gamma}{d_\gamma+t_\gamma}$
				\Comment $score \in [0, 1]$
				\State \Return $100\times score$
			\Else
				\State\Return 0
			\EndIf
		\EndFunction
	\end{algorithmic}
\end{algorithm}

The Time Distance Analyser uses the Time Analyser and Distance Analyser function which were defined as being $\Theta(|r|)$.
And then does a check and some arithmetic operations.
The limiting factor is the called functions so the complexity of this function must still be $\Theta(|r|)$.

\begin{algorithm}
	\caption{The Analyse Route pseudocode}
	\begin{algorithmic}
		\Require
		\Statex $scores(r, r') \rightarrow \mathbb{Z}$ : a matrix of scores between two routes
		\Statex
		\Procedure{AnalyseRoute}{$r, R$}
			\State $r_s\in r$ : the start point of route $r$
			\State $r_g\in r$ : the end point of route $r$
			\ForAll{$r'\in R$}
				\State $r'_s\in r'$ : the start point of route $r'$
				\State $r'_g\in r'$ : the end point of route $r'$
				\State $score(r', r)\gets$\Call{TimeDistanceAnalyser}{$r', r_s, r_g$}
				\State $score(r, r')\gets$\Call{TimeDistanceAnalyser}{$r, r'_s, r'_g$}
			\EndFor
		\EndProcedure
	\end{algorithmic}
\end{algorithm}

The final procedure takes in a route and a set of routes and calls the Time Distance Analyser two times for each route in $R$.
If we consider $\langle R \rangle$ to be the average length of all routes in $R$ and $|R|$ to be the number of routes in $R$.
Then the complexity would be $\Theta(|R|\times(\langle R\rangle + |r|))$.
$|r|$ could be represented within the average.
So the reduced complexity is $\Theta(|R|\times\langle R\rangle)$.
As the number of routes get greater the average route length gets harder to change between calculations and approaches therefore a constant value.
This means that as the set gets larger we approach this complexity $\Theta(|R|)$.

