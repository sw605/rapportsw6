\section{Analysis}
This section contains the analysis of app functionality, Android app development, ridesharing algorithms and communication with the aSTEP system.

%temporary title
\subsection{App functionality}
% Interviews and shit

% other shit...



\subsection{Android Development Platform}
When developing Android applications one of the first choices to consider is which API levels to target \cite{usesSDK}.
The manifest\todo{elaborate} requires to specify the API level\todo{elaborate} for three definitions: minimum, target and maximum. \todo{connect API level and SDK version}

The target and max versions requires few consideration to choose.
The "targetSdkVersion" reflects the version to which the app is developed and tested against, without enabling compatibility behaviors.
The "maxSdkVersion" reflects the highest API level that an app is designed to run.
Both of these are set to the highest API level available during the start of the development, which is API version 23.
\todo{add android dashboard figure}

The last API level which needs to be specified is the "minSdkVersion".
This entry specifies the lowest API level which an app supports.
A cause of this is that every device with an API level lower than the specified minSdkVersion will not be compatible with the app.
Supporting older API versions requires development efforts, typically implementing functionality though the Android Support Library \cite{androidSL}.
For this project we choose to target a higher API level for to simplify the development process.
At the time of writing, the platform version distribution sum of API levels greater than or equal to 21 is 35,3\%.
API version 21 is chosen as the minSdkVerison, the main reason for this choice is to minimize development time for user interface, as API level 21 added support for Android's new material design style \cite{android5API}. Refraining from further backwards compatibility allows the project focus to be on the functionality of the app. 

% algorithm analysis if necessary
\subsection{Algorithms for ridesharing}


% aSTEP system functionality analysis: communication with aSTEP
\subsection{Communication with \gls{astep}}\label{ssec:communicationwithastep}
To be able to utilize the \gls{astep} system, the communication protocol and its functions must be analyzed. 
The analysis is based on the current development and plan of the API and other functions.

% \gls{astep} organization - groups
The intended functions are established in collaboration with other project groups responsible for the \gls{astep} system side of the development. 
%We are mainly involved with the user management group and outdoor location based services groups.

% What does the \gls{astep} system look like?
The design of the \gls{astep} system is currently consisting of an API that apps and services can communicate with, and a backend with user management and location based services, which is stored in a database system.
However, the only relevant part for the \gls{rs} solution is the API, as the internal parts of the \gls{astep} core are administrated by other groups, and are not available through the API.

% What is stored in \gls{astep}
The \gls{astep} system will store information regarding location data, and basic user information. 
The data stored in the \gls{astep} system, relevant to this project solution, is:
\begin{itemize}
	\item Location data consisting of userID, routeID, GPS coordinates and timestamp.
	\item Username
	\item Password
\end{itemize}

The \gls{astep} user management system does not provide storage of data regarding contact information for \gls{astep} users.
The only information stored is a username and password to keep the \gls{astep} core as simple as possible.
Additional information that is required to make the app work as intended is each of the app groups own responsibility.
The \gls{astep} users are made to ensure the correct permissions are given to the correct user and so that the appropriate data is returned to each user.
An API call cannot be done without the user first being authenticated with a valid login.

% How to cummunicate with \gls{astep}
The communication with \gls{astep} is done through a REST API over HyperText Transfer Protocol, decided in agreement between the \gls{astep} project groups.
REST is an abbreviation of Representational State Transfer, and is a communication design often used in for HTTP-communication \cite{REST}.
Accordingly, the communication is performed by making queries to the \gls{astep} system. 
All communication must be done as a request from the device, to which the \gls{astep} server will respond.

% api functions
The currently available API functions at this stage of the development of \gls{astep} from user management and location services can be seen in Table \ref{tab:sprint2-api}.
The API is providing a POST-request under the name ``PostLocationDat'', as listed in \ref{tab:sprint2-api}, which is the current method to use when sending location data to the \gls{astep} system.
The call will accept location data as a coordinate consisting of longitude and latitude, a precision value, and a value representing time of day in milliseconds.

\begin{table}[!ht]
	\centering
	\begin{tabular}{ll}
		LBS & UM    \\
		\hline
		\begin{tabular}[t]{@{}l@{}}
			GetAllEntitiesInArea\\ GetAllEntitiesInTimePeriod\\ GetAllGroupMemebersLocationAndName\\ GetAllFriendsInArea\\ GetAllFriendsInRadius\\ GetAllGroupMembersInArea\\ GetAllGroupMembersInRadius\\ PostLocationData\end{tabular}
		&
		\begin{tabular}[t]{@{}l@{}}
			Create user\\ Get token\\ Update password\\ Edit privacy settings\\ Allow user2 to access user1's info\end{tabular}
	\end{tabular}
	\caption{Currently planned \gls{astep} API functions.}
	\label{tab:sprint2-api}
\end{table}


As both the \gls{astep} service and Android system have been analyzed it is possible to establish a set of requirements for the second sprint.

