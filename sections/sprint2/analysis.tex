\section{Analysis}
This section contains the analysis of app functionality, Android app development, ridesharing algorithms and communication with the aSTEP system.

%temporary title
\subsection{App functionality}
% Interviews and shit

% other shit...



\subsection{Android Development Platform}
When developing Android applications one of the first choices to consider is which API levels to target \cite{usesSDK}.
The manifest\todo{elaborate} requires to specify the API level\todo{elaborate} for three definitions: minimum, target and maximum. \todo{connect API level and SDK version}

The target and max versions requires few consideration to choose.
The "targetSdkVersion" reflects the version to which the app is developed and tested against, without enabling compatibility behaviors.
The "maxSdkVersion" reflects the highest API level that an app is designed to run.
Both of these are set to the highest API level available during the start of the development, which is API version 23.
\todo{add android dashboard figure}

The last API level which needs to be specified is the "minSdkVersion".
This entry specifies the lowest API level which an app supports.
A cause of this is that every device with an API level lower than the specified minSdkVersion will not be compatible with the app.
Supporting older API versions requires development efforts, typically implementing functionality though the Android Support Library \cite{androidSL}.
For this project we choose to target a higher API level for to simplify the development process.
At the time of writing, the platform version distribution sum of API levels greater than or equal to 21 is 35,3\%.
API version 21 is chosen as the minSdkVerison, the main reason for this choice is to minimize development time for user interface, as API level 21 added support for Android's new material design style \cite{android5API}. Refraining from further backwards compatibility allows the project focus to be on the functionality of the app. 

% algorithm analysis if necessary
\subsection{Algorithms for ridesharing}


% aSTEP system functionality analysis: communication with aSTEP
\subsection{Communication/collaboration with aSTEP}
% meta
To be able to utilize the aSTEP system, the communication protocol and functions must be analyzed. 
The analysis is based on the current development and planning of the API and other functions.

% aSTEP organization - groups
The status and intended functions are established in collaboration with the other project groups responsible for the aSTEP system side of the development. 
The main groups involved are the user management group and outdoor location based services group.

% What does the aSTEP system look like?
The current design of the aSTEP system is consisting of an API that apps and other external devices can communicate with, and a backend with user management and location based services.


% What is stored in aSTEP
The aSTEP system will store information regarding location data, and only basic information of users. 
The data stored in the aSTEP system, relevant to this project solution, is:
\begin{itemize}
	\item Location data, with userID, route, longitude, latitude and timestamp
	\item Username
	\item Password
\end{itemize}

The aSTEP user management system does not provide storing of data regarding the actual name, contact information nor profile picture for aSTEP users. 
%This affects the rideshare solution, as there must be an external unit to store the aforementioned attributes. 


% How to cummunicate with aSTEP
The communication with the aSTEP is done through a REST API over Hypertext Transfer Protocol.
Accordingly, the communication is performed on a request and response basis. 


% api functions
At the current stage of the development of aSTEP, the following API functions are available for user management and location services:


\begin{table}[]
	\centering
	\begin{tabular}{ll}
		UM & LBS    \\
		\hline
		\begin{tabular}[t]{@{}l@{}}
			GetAllEntitiesInArea\\ GetAllEntitiesInTimePeriod\\ GetAllGroupMemebersLocationAndName\\ GetAllFriendsInArea\\ GetAllFriendsInRadius\\ GetAllGroupMembersInArea\\ GetAllGroupMembersInRadius\\ PostLocationData\end{tabular}
		&
		\begin{tabular}[t]{@{}l@{}}
			Create user\\ Get token\\ Update password\\ Edit privacy settings\\ Allow user2 to access user1's info\end{tabular}
	\end{tabular}
	\caption{Currently planned aSTEP API functions.}
	\label{tab:sprint2-api}
\end{table}


