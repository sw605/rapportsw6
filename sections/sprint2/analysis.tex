\section{Analysis}
This section contains the analysis of app functionality, Android app development, \gls{rs} algorithms and communication with the aSTEP system.

%temporary title
\subsection{App functionality}
% meta shit
The app user interface is a low priority task during this project.
However, the app needs basic functionality, according to the requirements.

% User management
To be able to separate different routes, and to assign them to their respective users, there has to be a user management system, so that each user has its own ID.

% Ride matches
The user should also be informed with relevant information of ride matches, so that the user can make a decision of sharing a ride or not.

% User settings
The user of the application should also be able to adjust their settings.
The settings are preferences regarding different properties of ride sharing.
The user should be able to decide if it wants to get a ride or give a ride or both.
There are also properties of the user itself, like a description and a picture, that other users can access to decide if they want to share a ride with the named user.



% other shit...



\subsection{Android Development Platform}
When developing Android applications one of the first choices to consider is which API levels to target \cite{usesSDK}.
The ´´\textit{API Level is an integer value that uniquely identifies the framework API revision offered by a version of the Android platform.}'' according to Android Developers \cite{usesSDK}.
The distribution of the Android version levels are shown in \ref{fig:dashboard}.

´´\textit{The manifest file presents essential information about your app to the Android system, information the system must have before it can run any of the app's code.}'', according to Android Developers \cite{androidManifest}.
The manifest requires to specify the API level for three definitions: minimum, target and maximum. 
The target and max versions requires few consideration to determine.
The "targetSdkVersion" reflects the version to which the app is developed and tested against, without enabling compatibility behaviors.
The "maxSdkVersion" reflects the highest API level that an app is designed to run.
Both of these are set to the highest API level available during the start of the development, which is API version 23.

\begin{figure}[h!]
	\centering
	\includegraphics[width=0.7\textwidth]{figures/android-chart-march.png}
	\caption{Android version distribution, March 2016 \cite{androidDashboard}}
	\label{fig:dashboard}
\end{figure}

The last API level which needs to be specified is the "minSdkVersion".
This entry specifies the lowest API level which an app supports.
A cause of this is that every device with an API level lower than the specified minSdkVersion will not be compatible with the app.
Supporting older API versions requires development efforts, typically implementing functionality though the Android Support Library \cite{androidSL}.
For this project we choose to target a higher API level for to simplify the development process.
At the time of writing, the platform version distribution sum of API levels greater than or equal to 21 (Android Lollipop) is 35,3\%.
API version 21 is chosen as the minSdkVerison, the main reason for this choice is to minimize development time for user interface, as API level 21 added support for Android's new material design style \cite{android5API}. Refraining from further backwards compatibility allows the project focus to be on the functionality of the app. 

% algorithm analysis if necessary
\subsection{Algorithms for ridesharing}


% aSTEP system functionality analysis: communication with aSTEP
\subsection{Communication/collaboration with aSTEP}
% meta
To be able to utilize the aSTEP system, the communication protocol and functions must be analyzed. 
The analysis is based on the current development and planning of the API and other functions.

% aSTEP organization - groups
The status and intended functions are established in collaboration with the other project groups responsible for the aSTEP system side of the development. 
The main groups involved are the user management group and outdoor location based services group.

% What does the aSTEP system look like?
The current design of the aSTEP system is consisting of an API that apps and other external devices can communicate with, and a backend with user management and location based services.


% What is stored in aSTEP
The aSTEP system will store information regarding location data, and only basic information of users. 
The data stored in the aSTEP system, relevant to this project solution, is:
\begin{itemize}
	\item Location data, with userID, route, longitude, latitude and timestamp
	\item Username
	\item Password
\end{itemize}

The aSTEP user management system does not provide storing of data regarding the actual name, contact information nor profile picture for aSTEP users. 
%This affects the rideshare solution, as there must be an external unit to store the aforementioned attributes. 


% How to cummunicate with aSTEP
The communication with the aSTEP is done through a REST API over Hypertext Transfer Protocol.
Accordingly, the communication is performed on a request and response basis. 


% api functions
At the current stage of the development of aSTEP, the following API functions are available for user management and location services:


\begin{table}[]
	\centering
	\begin{tabular}{ll}
		UM & LBS    \\
		\hline
		\begin{tabular}[t]{@{}l@{}}
			GetAllEntitiesInArea\\ GetAllEntitiesInTimePeriod\\ GetAllGroupMemebersLocationAndName\\ GetAllFriendsInArea\\ GetAllFriendsInRadius\\ GetAllGroupMembersInArea\\ GetAllGroupMembersInRadius\\ PostLocationData\end{tabular}
		&
		\begin{tabular}[t]{@{}l@{}}
			Create user\\ Get token\\ Update password\\ Edit privacy settings\\ Allow user2 to access user1's info\end{tabular}
	\end{tabular}
	\caption{Currently planned aSTEP API functions.}
	\label{tab:sprint2-api}
\end{table}


% Sprint 2 krav
\subsection{Requirements for the second sprint}
%meta
Sprint two is the first of two sprints which have implementation as the primary focus. In this section the main problems to be solved will be presented.

%text
In this sprint the algorithm for comparing routes must be researched and developed. I should be done to an extend where it is possible to construct pseudo code of the algorithm, which should be ready to hand over to another group in the aSTEP system, one of the outdoor locations groups. Where they will start looking at implementing the algorithm on the aSTEP server.

User information such as username, password, and login token, should be analyzed to figure out where this information should be stored, and how it should be handled regarding communication between the application and the aSTEP server. 

A base for the functionality for the RideShare application should be implemented. This should consist of an working implantation of the Google Play Services and the application should be able to collect and store location data. The application should also have functionality to run as a background services, so data can be collected at all times.

Communication between the RideShare application and aSTEP server should also be researched and a implementation of the communication should be started in this sprint, ensuring the next sprint will have a base for communication. It should be considered how to handle storing of data collected and how much should be stored where, either on the device or on the aSTEP server.

It should be considered during the sprint, if the time is to generate some mock up location data related to mock up users. This data should be used in sprint three to test if the algorithm works as intended.