\section{Test}
The purpose of the test chapter is to document the examination of the current status and functionality of the implementation of the solution.
An informal test was performed on the app to assess the behavior and precision of the location gathering as a background service.

\subsection{Background service}
To assess the functionality of the background service, it was necessary to receive some kind of information, confirming that the service was actually running.
The background service was tested by making a simple IDE console output, that got printed each time the \texttt{jobScheduler} executed the \texttt{PendingIntent}.

In accordance with the implementation, the output appeared every 10 seconds, independent of the app being open, in the background or closed completely on the test device.
This proved that the background service was functioning as intended.

\subsection{Location gathering}
The location gathering was also performed in the test of the background service.
The location test was performed by trying to access the last known location in each scheduled \texttt{PendingIntent}, and to print the result in the debug log in the IDE.
The background service was executed regularly, but the location was only available once in a significant number of attempts.
The missing locations seemed to be caused by the \texttt{googleApiClient} not being connected, as the location gathering is performed in the \texttt{onConnected} method.
The missing locations seems to derive from the thread not existing long enough for the \texttt{googleApiClient} to connect to the services.

The implementation of the \texttt{jobScheduler} made it impossible for the \texttt{GoogleApiClient} to connect, hence not being able to assess the device location.
Additionally, the interval based location gathering, would be using battery consequently, regardless of the activity of the user. 
It is preferable to activate GPS location feature as little as possible, because GPS location collection is battery consuming.