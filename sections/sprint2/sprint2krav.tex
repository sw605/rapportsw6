%meta
Sprint two is the first of the two middle sprints that have implementation as the primary focus. In this section, the main issues to be solved in the current iteration will be presented, prioritized by the most important first.

% algorithm
\textbf{Route Matching Algorithm}\\
In this sprint, the algorithm for comparing routes must be and developed. 
It should be developed to an extent so that it is constructed as pseudocode of the algorithm, to be handed over to the outdoor \gls{astep} group, and be implemented in the \gls{astep} system.

\textbf{User Data}\\
User information such as username, password, and login token, must be analyzed to figure out where this information should be stored, and how it should be handled regarding communication between the application and the aSTEP server. 

\textbf{\gls{rs} App}\\
A base for the functionalities for the \gls{rs} application should be implemented. 
The implementation should include a version of the location collection.
The collector must include the Google Play Services and the app should be able to store the collected location data. 
The application should also have functionality to run as a background services, so that location data can be collected at all times.

\textbf{System Architecture and Communication}\\
Communication between the \gls{rs} application and \gls{astep} server should also be researched and an implementation of the communication should be initialized in this iteration, ensuring the next iteration will have a base for communication. 
It should be considered how to store the collected location data and how much should be stored where, either on the device or on the \gls{astep} server.

%\textbf{Mock Data}\\
%If development allows time for further work in this sprint it should be considered to generate mock up location data related to mock up users. 
%This data can be used in the third iteration to test if the algorithm works as intended.

As a set of requirements is decided the following chapter to be presented will be the design phase.