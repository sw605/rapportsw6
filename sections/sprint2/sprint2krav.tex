%meta
Sprint two is the first of two sprints which have implementation as the primary focus. In this section the main problems to be solved will be presented.

%text
In this sprint the algorithm for comparing routes must be researched and developed. I should be done to an extend where it is possible to construct pseudo code of the algorithm, which should be ready to hand over to another group in the aSTEP system, one of the outdoor locations groups. Where they will start looking at implementing the algorithm on the aSTEP server.

User information such as username, password, and login token, should be analyzed to figure out where this information should be stored, and how it should be handled regarding communication between the application and the aSTEP server. 

A base for the functionality for the RideShare application should be implemented. This should consist of an working implantation of the Google Play Services and the application should be able to collect and store location data. The application should also have functionality to run as a background services, so data can be collected at all times.

Communication between the RideShare application and aSTEP server should also be researched and a implementation of the communication should be started in this sprint, ensuring the next sprint will have a base for communication. It should be considered how to handle storing of data collected and how much should be stored where, either on the device or on the aSTEP server.

It should be considered during the sprint, if the time is to generate some mock up location data related to mock up users. This data should be used in sprint three to test if the algorithm works as intended.