\subsection{Communication/collaboration with aSTEP}
% meta
To be able to utilize the aSTEP system, the communication protocol and functions must be analyzed. 
The analysis is based on the current development and planning of the API and other functions.

% aSTEP organization - groups
The status and intended functions are established in collaboration with the other project groups responsible for the aSTEP system side of the development. 
The main groups involved are the user management group and outdoor location based services group.

% What does the aSTEP system look like?
The current design of the aSTEP system is consisting of an API that apps and other external devices can communicate with, and a backend with user management and location based services.
However, the only relevant part for the rideshare solution is the API.

% What is stored in aSTEP
The aSTEP system will store information regarding location data, and basic user information. 
The data stored in the aSTEP system, relevant to this project solution, is shown in Table \ref{tab:storedinastep}.

% Stored in aSTEP: storedinastep
\begin{table}[h]
	\centering
	\begin{tabular}{l|l}
		Object & Description and properties \\
		\hline
		Location & Longitude, latitude, timestamp, routeID, userID \\
		Username & Display and reference name of the user \\
		Password & The authentication key of a user
	\end{tabular}
	\caption{Properties stored in the aSTEP system.}
	\label{tab:storedinastep}
\end{table}

The aSTEP user management system does not provide storage of data regarding contact information nor profile picture for aSTEP users. 
%This affects the rideshare solution, as there must be an external unit to store the aforementioned attributes. 


% How to cummunicate with aSTEP
The communication with aSTEP is done through a REST API over Hypertext Transfer Protocol, decided in agreement between the aSTEP project groups.
REST is an abbreviation of representational state transfer, and is a communications protocol that utilizes HTTP for the create, read, update and delete functions \cite{REST}.
Accordingly, the communication is performed by making queries to the aSTEP system. 


% api functions
At the current stage of the development of aSTEP, the following API functions are available for user management and location services are listed in Table \ref{tab:sprint2-api}.
The relevant functions for the RideShare app are PostLocationData, to store the users' location in the aSTEP system, as well as the CreateUser, UpdatePassword, GetToken, EditPrivacySettings and access settings. The last ones are user management functions in aSTEP, where the names are self-explanatory, except GetToken which is the login function.


\begin{table}[]
	\centering
	\begin{tabular}{ll}
		LBS & UM    \\
		\hline
		\begin{tabular}[t]{@{}l@{}}
			GetAllEntitiesInArea\\ GetAllEntitiesInTimePeriod\\ GetAllGroupMemebersLocationAndName\\ GetAllFriendsInArea\\ GetAllFriendsInRadius\\ GetAllGroupMembersInArea\\ GetAllGroupMembersInRadius\\ PostLocationData\end{tabular}
		&
		\begin{tabular}[t]{@{}l@{}}
			Create user\\ Get token\\ Update password\\ Edit privacy settings\\ Allow user2 to access user1's info\end{tabular}
	\end{tabular}
	\caption{Currently planned aSTEP API functions.}
	\label{tab:sprint2-api}
\end{table}
