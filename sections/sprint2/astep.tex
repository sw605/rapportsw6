\subsection{Communication/collaboration with aSTEP}
% meta
To be able to utilize the aSTEP system, the communication protocol and functions must be analyzed. 
The analysis is based on the current development and planning of the API and other functions.

% aSTEP organization - groups
The status and intended functions are established in collaboration with the other project groups responsible for the aSTEP system side of the development. 
The main groups involved are the user management group and outdoor location based services group.

% What does the aSTEP system look like?
The current design of the aSTEP system is consisting of an API that apps and other external devices can communicate with, and a backend with user management and location based services.
However, the only relevant part for the rideshare solution is the API.

% What is stored in aSTEP
The aSTEP system will store information regarding location data, and basic user information. 
The data stored in the aSTEP system, relevant to this project solution, is:
\begin{itemize}
	\item Location data consisting of userID, routeID, longitude, latitude and timestamp
	\item Username
	\item Password
\end{itemize}
\todo{D: table with astep specifications?}

The aSTEP user management system does not provide storage of data regarding contact information nor profile picture for aSTEP users. 
%This affects the rideshare solution, as there must be an external unit to store the aforementioned attributes. 


% How to cummunicate with aSTEP
The communication with aSTEP is done through a REST\todo{reference} API over Hypertext Transfer Protocol.
Accordingly, the communication is performed by requests to aSTEP. 


% api functions
At the current stage of the development of aSTEP, the following API functions are available form user management and location services:


\begin{table}[]
	\centering
	\begin{tabular}{ll}
		LBS & UM    \\
		\hline
		\begin{tabular}[t]{@{}l@{}}
			GetAllEntitiesInArea\\ GetAllEntitiesInTimePeriod\\ GetAllGroupMemebersLocationAndName\\ GetAllFriendsInArea\\ GetAllFriendsInRadius\\ GetAllGroupMembersInArea\\ GetAllGroupMembersInRadius\\ PostLocationData\end{tabular}
		&
		\begin{tabular}[t]{@{}l@{}}
			Create user\\ Get token\\ Update password\\ Edit privacy settings\\ Allow user2 to access user1's info\end{tabular}
	\end{tabular}
	\caption{Currently planned aSTEP API functions.}
	\label{tab:sprint2-api}
\end{table}
\todo{reference this table. Add information about the functions.}

When using the aSTEP API it should be ensure to use the correct calls in the right way.
The API is providing a POST under the name0 "PostLocationData", as listed above, which is the method to use when sending location data to the aSTEP core. When using this POST it is important to ensure the correct parameters is used.
The call will accept location data as an GPS coordinate consisting of longitude and latitude, a precision value, and a value representing time of day in milliseconds.
This is important as it is one of the first calls that will be used in this sprint.
