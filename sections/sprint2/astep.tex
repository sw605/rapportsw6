\subsection{Communication with \gls{astep}}\label{ssec:communicationwithastep}
To be able to utilize the \gls{astep} system, the communication protocol and functions must be analyzed. 
The analysis is based on the current development and planning of the API and other functions.

% \gls{astep} organization - groups
The intended functions are established in collaboration with the other project groups responsible for the \gls{astep} system side of the development. 
We are mainly involved with the user management group and outdoor location based services groups.

% What does the \gls{astep} system look like?
The design of the \gls{astep} system is currently consisting of an API that apps and services can communicate with, and a backend with user management and location based services, which is stored in a database system.
However, the only relevant part for the \gls{rs} solution is the API, as the lower level of the \gls{astep} core is administrated by other groups.
% What is stored in \gls{astep}
The \gls{astep} system will store information regarding location data, and basic user information. 
The data stored in the \gls{astep} system, relevant to this project solution, is:
\begin{itemize}
	\item Location data consisting of userID, routeID, GPS coordinates and timestamp.
	\item Username
	\item Password
\end{itemize}

The \gls{astep} user management system does not provide storage of data regarding contact information for \gls{astep} users.
The only information stored is a username and password to keep the \gls{astep} core as simple as possible.
Additional information that is required to make the app work as intended is each of the app groups own responsibility.
The \gls{astep} users are made to ensure the correct permissions is giving to the correct user and thus the appropriate data is returned to each user.
An API call can not be done without the user first being authenticated with a valid login.

% How to cummunicate with \gls{astep}
The communication with \gls{astep} is done through a REST API over HyperText Transfer Protocol, decided in agreement between the \gls{astep} project groups.
REST is an abbreviation of REpresentational State Transfer, and is a communications design often used in for HTTP-communication\cite{REST}.
Accordingly, the communication is performed by making queries to the \gls{astep} system. 
All communication must be done as a request from the device, where the \gls{astep} server then will respond.


% api functions
At the current stage of the development of \gls{astep}, the API functions are available form user management and location services can be seen in Table \ref{tab:sprint2-api}
\begin{table}[!ht]
	\centering
	\begin{tabular}{ll}
		LBS & UM    \\
		\hline
		\begin{tabular}[t]{@{}l@{}}
			GetAllEntitiesInArea\\ GetAllEntitiesInTimePeriod\\ GetAllGroupMemebersLocationAndName\\ GetAllFriendsInArea\\ GetAllFriendsInRadius\\ GetAllGroupMembersInArea\\ GetAllGroupMembersInRadius\\ PostLocationData\end{tabular}
		&
		\begin{tabular}[t]{@{}l@{}}
			Create user\\ Get token\\ Update password\\ Edit privacy settings\\ Allow user2 to access user1's info\end{tabular}
	\end{tabular}
	\caption{Currently planned \gls{astep} API functions.}
	\label{tab:sprint2-api}
\end{table}

When using the \gls{astep} API it should be ensure to use the correct calls in the right way.
The API is providing a POST-request under the name ``PostLocationDat'', as listed in \ref{tab:sprint2-api}, which is the method to use when sending location data to the \gls{astep} core.
When using this POST it is important to ensure the correct parameters is used.
The call will accept location data as an coordinate consisting of longitude and latitude, a precision value, and a value representing time of day in milliseconds.