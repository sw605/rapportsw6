\subsection{Communication/collaboration with aSTEP}
% meta
To be able to utilize the aSTEP system, the communication protocol and functions must be analyzed. 
The analysis is based on the current development and planning of the API and other functions.

% aSTEP organization - groups
The status and intended functions are established in collaboration with the other project groups responsible for the aSTEP system side of the development. 
The main groups we are involved with are the user management group and outdoor location based services groups.

% What does the aSTEP system look like?
The design of the aSTEP system at this point is consisting of an API that apps and services can communicate with, and a backend with user management and location based services, which is stored in a database system.
However, the only relevant part for the \gls{rs} solution for us is the API, as the lower level of the aSTEP core is administrated by other groups.

% What is stored in aSTEP
The aSTEP system will store information regarding location data, and basic user information. 
The data stored in the aSTEP system, relevant to this project solution, is:
\begin{itemize}
	\item Location data consisting of userID, routeID, a set of GPS coordinates and timestamp.
	\item Username
	\item Password
\end{itemize}

The aSTEP user management system does not provide storage of data regarding contact information for aSTEP users.
The only information stored is a username and password to keep the aSTEP core as simple as possible.
Additional information that is required to make the app work as intended is each of the app groups own responsibility.
The aSTEP users is made to ensure the correct permissions is giving to the correct user, to ensure the appropriate data is returned to each user.
An API call can not be done without the user first being authenticated with a valid login.

% How to cummunicate with aSTEP
The communication with aSTEP is done through a REST API over Hypertext Transfer Protocol.
Accordingly, the communication is performed by requests to aSTEP.
All communication must be done as a request from the device, where the aSTEP server then will respond.


% api functions
At the current stage of the development of aSTEP, the following API functions are available form user management and location services:
\begin{table}[]
	\centering
	\begin{tabular}{ll}
		LBS & UM    \\
		\hline
		\begin{tabular}[t]{@{}l@{}}
			GetAllEntitiesInArea\\ GetAllEntitiesInTimePeriod\\ GetAllGroupMemebersLocationAndName\\ GetAllFriendsInArea\\ GetAllFriendsInRadius\\ GetAllGroupMembersInArea\\ GetAllGroupMembersInRadius\\ PostLocationData\end{tabular}
		&
		\begin{tabular}[t]{@{}l@{}}
			Create user\\ Get token\\ Update password\\ Edit privacy settings\\ Allow user2 to access user1's info\end{tabular}
	\end{tabular}
	\caption{Currently planned aSTEP API functions.}
	\label{tab:sprint2-api}
\end{table}
\todo{reference this table. Add information about the functions.}

When using the aSTEP API it should be ensure to use the correct calls in the right way.
The API is providing a POST-request under the name ``PostLocationDat'', as listed in \ref{tab:sprint2-api}, which is the method to use when sending location data to the aSTEP core.
When using this POST it is important to ensure the correct parameters is used.
The call will accept location data as an coordinate consisting of longitude and latitude, a precision value, and a value representing time of day in milliseconds.
