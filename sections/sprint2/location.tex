\subsection{Location}
As documented in the analysis in Section \ref{ssec:loctrack} the choice of activity recognition and location gathering was the Google Play Services.

In the following test some values will be decided, in concern to the design of the location tracking. 
These values are yet uncertain and might be adjusted later as more knowledge of the practical implication of these are gathered. 

Activity recognition should be polled continuously to check whether a user is registered as in vehicle.
When a user is in a vehicle with a probability level above 80\%, a location request must be be performed.

The application must be logged into and running in the background while the device is turned on to be able to differ between the users activity at all times and collect data. 
The interval for collecting location data from the location services will be set to every two minutes, while in a vehicle, to reduce energy consumption. 
Every location collected during the driving activity must be appended to a list that represents an entire route when the driving activity ends.

When the activity change from vehicle to any other activity, the application will send the list of locations to the \gls{rs} server, where it will take over the processing. 