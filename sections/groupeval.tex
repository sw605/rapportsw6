\subsection{Multi-group Project Reflection and Evaluation}
Through this semester, there has been a high focus on collaboration between the groups in this multi-group project.
The collaboration effort consisted of two main parts: a weekly meeting where updates from all groups were made, and shared group rooms.
The factors made it possible to discuss important decisions, such as the structure of the \gls{astep} and requirements from the different groups, without scheduling a meeting elsewhere.
On the other side is community that sprouted from the shared group rooms where most was open and ready to help if needed.
If one group needed to speak to another group, then they would quickly arrange that over Slack, that was the selected online communication platform for the semester project.
Slack was a good choice as it provided great group structure and bot integrations, such as meeting reminders. 
The Slack platform enabled us to resolve small questions and to arrange meetings that helped in the collaboration effort.

We had collaborations with a few groups such as the Friend Finder group about the user interface aesthetics of the apps, to achieve a similar feel between the \gls{astep} apps, and the user management group about what user specific user features that were needed.
Our tightest collaboration, however, was with the outdoor location based service (OD) group, as we developed and implemented our route matching algorithm for them to implement.
During the project, we had several meetings with them where we would discuss what we would need from the API and how implementation should be done to meet our requirements.
In return they were concerned about performance and reducing very specialized implementations for apps in the \gls{astep} core as the core should be as generalized as possible.
We later on participated in an API test with OD, where they were interested in whether we would understand and be able to use the API.

We did not have as many collaborations as some of the other groups, as we only requested functionality and the \gls{astep} core would then implement it.
%Still, we experienced a good working environment because the weekly meetings meant that there was a certain amount of work that we were committed to do.\todo{what does this mean? - Bjørn}
But the collaboration we did have with the other groups were rewarding as we exchange ideas and thoughts.

Since this was a startup of a new project, there was confusion and frustration early in the period that had to be resolved.
The lack of structure and detailed agreements sometimes resulted in a negative attitude towards specific people or groups who made small mistakes, or did not comply with the other groups expectations.
The issues could have been avoided by having a dedicated scrum master for the overall project.
We were already developing in an agile fashion so it would not be to hard to implement.
In general, we feel that a better management and direction of the overall project was lacking.
A possible solution could be stricter guidelines from the project leader, or to have some sort of administrative group to delegate responsibilities.
