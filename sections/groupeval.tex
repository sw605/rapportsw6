\subsection{Multi-group Project Reflection and Evaluation}
This semester, there has been a high focus on collaboration between the groups in this multi-group project.
The collaboration effort consisted of two important parts: a weekly meeting where updates from all groups were made, and shared group rooms.
The factors made it possible to discuss important decisions, such as the structure of the \gls{astep} and requirements from the different groups, without scheduling a meeting elsewhere.
On the other side is community that sprouted from the shared group rooms where most was open and ready to help if needed.
If one group needed to speak to another group about a certain topic, then they would quickly arrange that over our online communication tool; Slack.
Slack was a good choice as it provided great role structure and provide bot integrations, such as meeting reminders.

This enabled us to solve small question and arrange indept meetings which helped a lot in the collaboration effort.

We had collaborations with a few groups such as the Freind Finder group about the color theme of the apps to get a similar feel between the \gls{astep} apps, and the User Management group about what user specific features were needed.

Our tightest collaboration was with the outdoor location based service (OD) group, as we developed and implemented our primary algorithm for them to implement.
During the project, we had several meetings with them where we would discuss what we would need from the API and how implementation should be done to meet our needs.
In return they were concerned about performance and reducing very specific implementations\todo{very specific implementations is not very specific - Corlin} in the \gls{astep} core.
We later on participated in an API test with OD, where they were interested in whether we would understand and be able to use the API.

We did not have as many collaborations as some of the other groups had as we only requested functionality and the core would then implement it.
It would have been better if we had some potential costumers to request features of us.
But even then did we experience a better working environment because the weekly meetings ment that there were a certain amount of work that we were committed to do.
We also found communicating with the other groups very rewarding as we could share ideas and toughs.

Since this is a new project there was a lot of confusion and frustration in the start which we had to overcome.
This resulted some times in a negative attitude towards people or groups who made small mistakes.
This could be overcome by having a dedicated scrum master for the overall project.
We were already developing in an agile fashion so it would not be to hard to do.
In general, we feel that a better management and direction of the overall project was lacking.
