\section{Future Work}\label{sec:future}
This section contains description of work that could be done in future semesters, if the \gls{rs} system should be refined and further advanced.

\subsection{User Acceptance}
As of this semester our system is solely developed based on our analysis of the interest and development in ridesharing solutions.
Before further development it would be ideally to actual test whether the expected interest in the solutions actual exist.
In addition validation of a marked for the solution, users interviews or similar research could also be used to evaluate the current state of the system and which changes or improvements users or potential users would prefer.

\subsection{App Functionality}
Currently, the app does not take users who are not driving into account.
It could be useful to consider users who are biking or walking to work, and give them the opportunity to get a ride match.
This could also lead to improving the algorithm, as to decide who can be matched against whom.
For instance it would not make sense to see if someone who is riding a bike could pick up other users.

When using the current version of the algorithm it is implemented so that two users are matched with each other.
But as it is now, only one of the users will receive the information about the match.
The user receiving the notification is the one who have the shortest detour to pick up the other, so he or she can decide if it is worth it. 
In the future a change could be made so that users who wants to be picked up could also see the matches and request a ride.

\subsubsection*{Registration of Users}
The edges described in Section \ref{subsec:usermanagement} was not implemented as they were not finished in the API.
Implementation could be done during user registration with a few API calls to \gls{astep}.
It could also be possible to integrate login possibilities such as Google or Facebook as alternative login methods and user info providers.

Another feature which can become important in the future is allowing users to use the same \gls{astep} user for different applications.
This could be done by utilizing tokens and improving how they are handled in the \gls{astep} system.

\subsubsection*{User Interface}
Since the user interface was not a priority of this project, it could benefit from an redesign.
This redesign could include a more consistent visual theme between the views, and functionality that makes the app more easier to use.
Profile pictures for accounts could make  users more trustworthy to users whom they match with.
More user information could also be displayed when tapping on a user match.
Displaying details about a route that a user was matched with could be represented on a map, where a seemingly natural integration is with Google Maps.

Improvements is needed handling and displaying error messages users, which currently are mostly non-informative or non-existent, because many of those messages were written for development purposes.

\subsection{Server}
Some user interface changes will require server support.
Currently, there is no support for storing profile pictures.

It could also be an improvement if the server could send push notifications with new routes matches to a user's app when detected, instead of the current implementation where the app request the server for updates when the app is opened.

% Messenger service
The current option for establishing contact between users is by phone numbers.
The user's phone number is sent to strangers also using the app, which could be unwanted by users.
A better solution would be a messaging system, integrated in the app itself or another similar way of users whom matched, to communicate.

\subsection{Algorithm}
For the algorithm several refinements can be made.
Some actual user test of the algorithms accuracy and some more formal testing would give a better basic.
Both in regards to deciding which parts of the algorithm could use improvement most, as well as gathering some information about which parameters are most important to user when considering possible matches.

The input parameters of the algorithm could also be evaluated or expanded upon.
As mentioned in \ref{s:s1anal} user could specify preferences of the people whom they will be matched against, i.e. gender, or if a person smokes.

One possible thing to improve upon is the search space for matches.
The algorithm does not use any filtering of routes other than a simple time comparison.
Additional filtering could be done by using geo-fencing.
Another less complex solution could be to truncate the coordinate in to a lower accuracy location.
By lowering the number of coordinate decimals in the accuracy, a larger margin of precision is expressed, with the consequences of making a `grid' as can be used for filtering.
Truncating the location values down to two decimals as described by \citet{gpsdecimal}, for example, yields an accuracy of 1.1 kilometers.

One of the most important feature improvements to the algorithm, is to consider maps, speed limits, traffic data and terrain when calculating detours.
This feature could increase the accuracy of the estimated detour time and distance, and hence yield more accurate scores.
It could be implemented by utilizing the Google Maps API.

The above documented possible future work does not reflect all possible directions which the project could take, but rather it is a reflection of the thoughts and discussions which our work on the system has fostered.