\section{Future Work}
This section contains the features that were thought of during the project period, but were not implemented because of time constraints or disregarded to maintain a consistent focus for the project.

\subsection{Registration of users}
Because having connections between the users of the app and the \gls{rs}'s \gls{astep} user does not make any difference in the current version of \gls{astep}, the edges described in Section ?? \todo{} was not implemented.
This should be done during user registration with a few API calls.
It might be possible to also include a way to intergrate facebook as a alternative login method.

Another feature that is going to be important in the future is to allow users to use the same \gls{astep} user for multiple applications.
This can be done by introducing a two step registration, with the first step being logging in to \gls{astep} or creating a new \gls{astep} user.
And the second step being creating a new \gls{rs} user.
How this is achived is up to the group implementing it, but it could be done though an altered user interface.

\subsection{User interface}
And when talking about user interface.
It is stated multiple times that the focus of this project was never the interface, and it could therefor use an overhaul.
This overhaul would include a more consistent design between views and possibly a number of functionality that makes the app more enjoyable to use.

This could be profile pictures for matches, making the user more confident in that the person they are driving with is the same they matched with.
Another feature would be to display additional information when a user tapped on a match, a new activity would start which shows formation about the match.
This information could include other people that often travels with the user and details about the others route which were matched with.
One way to display details about the route that were matched witch would be to plot it into some sort of Google Maps intergration.

The app also does not take into account the users which are not driving.
This feature would require a rework of the internal route system in the \gls{astep} system.
Another thing that could be improved is the error messages displayed to the user, which currently is mostly non-informative or non-existent.

\subsection{Server}
To make the changes to the user interface a rework of some of the parts in the server would also need to be reworked.
Currently, there is no way to store profile pictures, and only meta data about routes are stored.
This means that there is currently no way to implement profile pictures, and there is not enough information in the meta data to plot the routes on a map.



%Register users på den rigtige måde med edges
%App
%	UI
%		Proper error handling/messages to user
%		App instruction pop-ups on first start
%	map view af ruter der er matched med
%	Vis profilbilleder
%Flyt overførslen af router over til body istedet for URL
%RSS
%	Gem profilbilleder
%	Push route matches
%Astep aka Algorithm
%	Display matches both ways?
%	Optimizations (d-trees?)
%	Grid/cell structure
%	Map integration (actual road distances and speed limits)
% Tag højde for folk der kører og ikke kører
