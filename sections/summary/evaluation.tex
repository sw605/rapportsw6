\chapter{Summary}
This summary chapter contains the reflection, conclusion and further work.
First we will reflect upon the project, with regards to both the internal group work and the multi-group experience.
Thereafter, a conclusion is made for the project and solution.
Last, we describe our thoughts about potential of the solution and possible future work.

\section{Reflection}
This section contains our reflection of the different project aspects.
First our own project reflection is reflected upon, then the multi-group project is reflected upon and evaluated.

\subsection{Project Reflection}
% project conten and system design/implementation
The system design was influenced by the \gls{astep} progression during the development as the system relied on \gls{astep}.
Throughout the project, there were multiple changes in the \gls{astep} system, and they had different impacts on our project.
Most significant was the change from \gls{astep} storing supplemental user information, such as e-mail address, to not storing anything other than username and password.
The system design could have been better and there would be more time for implementation if the \gls{astep} system design had a final and established design from the beginning.

% project process and work distribution
The implementation was successful and dynamic, adjusting to changes made in the \gls{astep} project and our own tests performed in the iterations, in accordance with the iterative method.
All group members were included in every part of the project.
Each member had a possibility to discuss the individual parts, despite being assigned to another part during that time period.
The tasks were evenly distributed on the group members, and more complex tasks were assigned to multiple members, where necessary.

% project work load
We were ambitious with the extent of the project, therein objectives regarding custom user settings of having a car or not, despite the project leader Bin Yang proposed low ambition levels and just making a working basis.
Our ambition level is reflected in the requirement specification, and as seen in the acceptance test in Section \ref{sec:s4test}, only the 'must have' requirements are fulfilled.
The workload during the project has been high while there has been a joint attitude of making the solution as requirement-fulfilling as possible.
The ambition level and the established requirement specification together with unexpected task completion durations have caused delays during the sprints so that actual deadlines were sometimes not met.
This happened during sprint 2 and 3, where our internal deadline would extend two weeks past the multi-group project deadlines.
The missed deadlines had no effect on the \gls{astep} project and only made our own scheduling harder to meet.

% realism comment
The solution development could have been more realistic if we had potential costumers to request features of us.
This would apply more pressure to and reliance on the group collaborations and flow of information within the \gls{astep} project.

% boom?
The \gls{rs} app is a functional prototype, and is seen as a success since it utilized the \gls{astep} system.

\subsection{Multi-group Project Reflection and Evaluation}
Through this semester, there has been a high focus on collaboration between the groups in this multi-group project.
The collaboration effort consisted of two main parts: a weekly meeting where updates from every group were made, and shared group rooms.
The factors made it possible to discuss important decisions, such as the structure of the \gls{astep} and requirements from the different groups, without scheduling a meeting elsewhere.
On the other side is the community that sprouted from the shared group rooms where most was open and ready to help if needed.
If one group needed to speak to another group, then they would quickly arrange that over Slack, that was the selected online communication platform for the semester project.
Slack was a good choice as it provided great group structure and bot integrations, such as meeting reminders. 
The Slack platform enabled us to resolve small questions and to arrange meetings that helped in the collaboration effort.

We had collaborations with a few groups such as the Friend Finder group about the user interface aesthetics of the apps, to achieve a similar feel between the \gls{astep} apps, and the user management group about what user specific user features that were needed.
Our tightest collaboration, however, was with the outdoor location based service (OD) group, as we developed and designed our route matching algorithm for them to implement.
During the project, we had several meetings with them where we would discuss what we would need from the API and how implementation should be done to meet our requirements.
In return they were concerned about performance and reducing very specialized implementations for apps in the \gls{astep} core as the core should be as generalized as possible.
We later on participated in an API test with OD, where they were interested in whether we would understand and be able to use the API.

We did not have as many collaborations as some of the other groups, as we only requested functionality and the \gls{astep} core would usually then implement it.
But the collaboration we did have with the other groups were rewarding as we exchange ideas and thoughts.

Since this was the startup of a new project, there was confusion and frustration early in the period that had to be resolved.
The lack of structure and detailed agreements sometimes resulted in a negative attitude towards specific people or groups who made small mistakes, or did not comply with the other groups expectations.
The issues could have been avoided by having a dedicated scrum master for the overall project.
We were already developing in an agile fashion so it would not be to hard to implement.
In general, we feel that a better management and direction of the overall project was lacking.
A possible solution could be stricter guidelines from the project leader, or to have some sort of administrative group to delegate responsibilities.

