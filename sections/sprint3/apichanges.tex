\subsection{Changes to the aSTEP API}\label{s3:apianal}
In the period since the previous sprint the API to the \gls{astep} have been updated with new functionality both regarding to user management and the location based services.
The LBS are split into both indoor locations and outdoor locations calls, in this report only the outdoor service will be considered since it is the only relevant. 

The list of all the API functionality as of the third sprint can been seen on Table \ref{tbl:s3api}.

\begin{table}[!ht]
	\centering
	\begin{tabular}{ll}
		Outdoor location services & User Management    \\
		\hline
		\begin{tabular}[t]{@{}l@{}}
			Get user past location\\ Get user past locations area\\ Get user past Locations radius\\ post user current location\\ Post user current route\\ Get all new route matches\\ Get all locations friends\\ Get all users in area \\ Get locations friends \\ Get subset of users area \\ Get users outside area \\Get subset of users radius \\ Get locations friend timestamp\end{tabular}
		&
		\begin{tabular}[t]{@{}l@{}}
			Create group \\ Get administrated groups\\ Add administration\\ Remove administration \\ Get invited users \\ Invite user to group \\ Revoke group invitation \\ Get members of a group \\ Remove user from group \\ Create user \\ Get groups that a user is invited to \\ Decline group invitation \\ Get groups that user is member of \\ Join group \\ Leave group \\ Get outdoor-user\\ Request outdoor user \\ Delete outdoor-user \\ Validate outdoor-user\\ Change password\\ Get token \\ Issue token\end{tabular}
	\end{tabular}
	\caption{Currently planned aSTEP API functions.}
	\label{tbl:s3api}
\end{table}

When comparing these API functionalities to the functionalities in sprint 2 it can be seen that the the API has evolved a lot and many new functions is available.
For the location based service the \gls{astep} system now offers different get requests which regards to a users location history, nearby users and users based on groups or friends.
The services offered by the user management is also expanded a lot since previous sprint. 
Mainly UM have implemented the concept of groups which entails a lot of new API calls. 
The concept of  groups are implemented as an access control for which users have access to other users data.
The design of our system will need to reflect these changes, which will be further analyzed in the following sections.  
The particulars of the different API calls will further explained in following sections as they are used.