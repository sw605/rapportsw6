%meta text
This section contains an overview of the implementation of the API calls utilized to communicate with \gls{astep}.

In the previous design phases it was decided what API calls should be implemented in the app.
A specified list of calls only to be performed on the app can be seen in Table \ref{tab:asteprequests}.

\begin{table}[h]
	\centering
	\scriptsize
	\begin{tabular}{l l l}
		Path & Method & Description\\\midrule
		/locations/outdoor/\{username\}/routes & POST & Send a route to the \gls{astep} server\\
		/users & POST & Create a new user\\
		/users/\{username\}/token & GET & Get the current valid token for a user\\
		/users & POST & Invalidate old token and get new\\
		/users/\{username\}/token & POST & Invalidates previous token for a user\\
	\end{tabular}
	\caption{Caption}
	\label{tab:asteprequests}
\end{table} 

The API calls are implemented using a \texttt{HttpURLConnection} class, which is a \texttt{URLConnection} class for HTTP specific data transfer over the web.
\texttt{HttpURLConnection} may be used to send and receive streaming data whose length is not known in advance.
This is needed as every API call varies in size.
E.g., the API call to post a route will be the one that varies the most in size as routes differ in size, based on the number of locations and their accuracy.


Using an object orientated structure, each API call is implemented to overwrite a \texttt{helperClass} which will handle the inputs, such as username, password, token, and routes.
The inputs are then transformed to a part of an URL which would fit each of the relevant API calls.
The \texttt{helperClass} is then parsed to the \texttt{HttpURLConnection} which will use a string builder to create the final URL from the API base URL, \texttt{http://astep.cs.aau.dk:80/api/}.
Lastly, the \texttt{HttpURLConnection} is trying to perform the API call from where response codes are handled.

It should be noted that passwords are not in the URL part of the API call, as these will be logged on the \gls{astep} server. 
This would be a major security issue.
Instead password and tokens are parsed as part of the body, which is used by REST. 
This will hide the content from the URL and instead be transfered in an \texttt{OutputStream} encoded in raw bytes. 

On two examples of API calls can be seen in List \ref{list:example}.
The post route example is simplified to reduce size and make it better explainable.

\todo{fix references and captions}

\begin{enumerate}
\label{list:example}
\item POST new user
\begin{enumerate}
\item URL: http://astep.cs.aau.dk:80/api/users?username=USERNAME
\item CURL: curl -X POST --header 'Content-Type: application/json' --header 'Accept: application/json' -d 'PASSWORD' 'http://astep.cs.aau.dk:80/api/users?username=USERNAME'
\end{enumerate}
\item POST route to \gls{astep}
\begin{enumerate}
\item URL: http://astep.cs.aau.dk:80/api/locations/outdoor/outdoor/routes?authorization=TOKEN\&route=longitude,latitude,precision,timestamp;\&route=longitude,latitude,precision,timestamp
\&distance\_weight\&time\_weight\&largest\_acceptable\_detour\\
\_length=146\&acceptable\'httptime\_difference=32
\item CURL: curl -X POST --header 'Content-Type: application/json' --header 'Accept: application/json' 'URL from above'
\end{enumerate}
\end{enumerate}

As seen in Create user in List \ref{list:example}, the username is parsed in the URL while the password is included in a header.
On the other hand, the post route is only parsed in the URL as there is no information that needs to be hidden.
Every decision made regarding parsing data from the apps to the \gls{astep} server is decided by the individual \gls{astep} API groups.
