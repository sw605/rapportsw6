%meta text
An overview of the implementation of the API calls to \gls{astep} will be described in the following chapter.

In previously design phases it were decided what API calls should be implemented in the app.
On \ref{tab:asteprequests} a specified list of calls only to be performed on the app can be seen.

\begin{table}[h]
	\centering
	\scriptsize
	\begin{tabular}{l l l}
		Path & Method & Description\\\midrule
		/locations/outdoor/\{username\}/routes & POST & Send a route to the \gls{astep} server\\
		/users & POST & Create a new user\\
		/users/\{username\}/token & GET & Get the current valid token for a user\\
		/users & POST & Invalidate old token and get new\\
		/users/\{username\}/token & POST & Invalidates previous token for a user\\
	\end{tabular}
	\label{tab:asteprequests}
	\caption{}
\end{table} 

The API calls are implemented using a \texttt{HttpURLConnection} class, which is a \texttt{URLConnection} class for HTTP specific data transfer over the web.
 \texttt{HttpURLCOnnection} may be used to send and receive streaming data whose length is not known in advance.
 This is needed as every API call varies in size.
 Looking at the API call to post a route, it will be the one that varies in size the most as every routes will differ in size.



Using a object orientate structure, each API call is implemented to overwrite a \texttt{helperClass} which will handle the inputs, such as username, password, token, and routes.
The inputs is then transformed to a part of an URL which would fit each of the relevant API calls.

The helperClass is then parsed to the \texttt{HttpURLConnection} which will use a string builder to create the final URL from the API base URL, \texttt{http://astep.cs.aau.dk:8080/api/}.
Lastly the \texttt{HttpURLConnection} is trying to perform the API call, from where response codes are handled.

It should be noted that passwords are not in the URL part of the API call, as these will be logged on the \gls{astep} server. 
This would be a major security issue.
Instead are password and tokens parsed as part of the body, which is used by REST. 
This will hide the content from the URL and instead be transfered in an \texttt{OutputStream} encoded in raw bytes. 

On two examples of API calls can be seen on \ref{list:example}.
Post route is simplified to reduce size and make it easier to explain.

\begin{enumerate}
\label{list:example}
\item POST new user
\begin{enumerate}
\item URL: http://astep.cs.aau.dk:8080/api/users?username=USERNAME
\item CURL: curl -X POST --header 'Content-Type: application/json' --header 'Accept: application/json' -d 'PASSWORD' 'http://astep.cs.aau.dk:8080/api/users?username=USERNAME'
\end{enumerate}
\item POST route to \gls{astep}
\begin{enumerate}
\item URL: http://astep.cs.aau.dk:8080/api/locations/outdoor/outdoor/routes?authorization=TOKEN\&route=longitude,latitude,precision,timestamp;\&route=longitude,latitude,precision,timestamp
\&distance\_weight\&time\_weight\&largest\_acceptable\_detour\\
\_length=146\&acceptable\'httptime\_difference=32
\item CURL: curl -X POST --header 'Content-Type: application/json' --header 'Accept: application/json' 'URL from above'
\end{enumerate}
\end{enumerate}

As seen on Create user in \ref{list:example}, the username is parsed in the URL while the password is included in a header.
On the other hand is the post route only parsed in the URL as there are no information that needs to be hidden.
Every decision made around parsing the data from the app to the aSTEP server is decided by the individual aSTEP API groups.


%hey

%farvel