\section{Implementation}
The implementation section contains descriptions of selected parts of the implementation performed in sprint 3.

The main focus in this implementation is the location gathering service, which will record locations continuously while the user is driving, but not otherwise. 

\subsection{Location Gathering Background Service}
% metatext
These paragraphs will describe the contents of the implementation of the background location gathering.
The new implementation is founded on the poor results of the location gathering method implemented in sprint 2, Section \ref{section:locationgathering}, which was not working as intended.
Code examples in this section will not include potential logging and debugging lines as for better readability and they do not influence the functionality of the code.

\textbf{Implementation plan}\\
According to the design, Section \ref{section:s3locgath}, the locations should first be requested when the user is driving.
Therefore, the implementation starts by instantiating an activity recognizer.
The activity recognizer can thereafter check the received activity, and begin location gathering if the activity is recognized as driving with a confidence above 80\%.
When the confidence is lower than the threshold for a period of time, the driving activity can be considered finished, and the route must be stored and later transmitted to the \gls{astep} system.


% activityRecognitionInstantiation
\textbf{Activity recognition instantiation}\\
The location gathering should first detect that the device is in a driving activity and then begin recording locations, according to the design described in Section \ref{ssec:LocationGatherBGS}.
To achieve the functionality, the app must connect to a Google API client and subscribe to activity recognition.
The code for subscribing to the activity recognition is shown in Listing \ref{activityRecognitionInstantiation}.
The activity recognition can thereafter check that the detected activity is \texttt{inVehicle}, and accordingly activate the location gathering.

\begin{lstlisting}[language=Java, label=activityRecognitionInstantiation, caption=Initialization of activity recognition.]
if(activityDetectionBroadcastReceiver == null) {
	activityDetectionBroadcastReceiver = 
		new ActivityDetectionBroadcastReceiver(googleApiClient, this);
		
	LocalBroadcastManager.getInstance(this).registerReceiver(
		activityDetectionBroadcastReceiver, 
		new IntentFilter("fapptory_inc.rideshare.BROADCAST_ACTION"));
}
\end{lstlisting}

The \texttt{activityDetectionBroadcastReceiver} is an instantiation of the\\ \texttt{ActivityDetectionBroadcastReceiver} class, utilizing the \texttt{googleApiClient} and \texttt{this} which in this case references the current \texttt{MainActivity} instance to receive detected activities.

The broadcast receiver is then applied by registering to the \gls{rs} app's\\ \texttt{BROADCAST\_ACTION}, so that the \texttt{activityDetectionBroadcastReceiver} only handles broadcasts sent by \gls{rs} and not by other apps.


% Activity Detection Broadcast Receiver
\textbf{Activity Detection Broadcast Receiver}\\
The main operations in regards to activity detection and location gathering are done by the activity detection broadcast receiver.

When the \texttt{activityDetectionBroadcastReceiver} receives a RideShare broadcast, the detected activities in the broadcast is stored in the \texttt{ArrayList<DetectedActivity> detectedActivities}.
The \texttt{detectedActivities} is iterated over, shown in Listing \ref{activityRecognitionIteration} to find the driving activity \texttt{DetectedActivity.IN\_VEHICLE}.
The \texttt{startLocationUpdates()}, which is described later, is called if the confidence is above the threshold and if locations are not currently being recorded.
Line 8 controls the location cache in case a drive is stopped temporarily and starts again, appending the cached route to the actual route.

\begin{lstlisting}[language=Java, label=activityRecognitionIteration, caption=Iteration over received list of activity recognition.]
for (DetectedActivity da: detectedActivities){
	if(da.getType() == DetectedActivity.IN_VEHICLE){
		if(da.getConfidence() >= ACTIVITY_CONFIDENCE_VALUE){
			if(!isCurrentlyCollectingLocations) {
				startLocationUpdates();
				isCurrentlyCollectingLocations = true;
			}
			if(potentialStopDrivingActivity && potentialAstepRoute.size() > 0){
				astepRoute.addAll(astepRoute.size(), potentialAstepRoute);
				potentialAstepRoute.clear();
			}
			lastDetectedVehicleActivity = System.currentTimeMillis();
			potentialStopDrivingActivity = false;
			drivingActivityDetected = true;
		}
	}
}
\end{lstlisting}

% Location Updates Request
\textbf{Location Updates Request}\\
The \texttt{startLocationUpdates()}, shown in Listing \ref{startLocationUpdates}, is called to gather locations regularly.
The location updates request is based on the Google API client instantiated in \texttt{MainActivity} and a \texttt{locationRequest}.
The \texttt{locationRequest} is defined with properties regarding location update interval, fastest interval, and location accuracy priority, respectively set to 60 seconds, 60 seconds and \texttt{PRIORITY\_HIGH\_ACCURACY}.

\begin{lstlisting}[language=Java, label=startLocationUpdates, caption=Start location updates functions.]
public void startLocationUpdates(){
	if(/* Omitted: Check permissions */) {
		LocationServices.FusedLocationApi
			.requestLocationUpdates(googleApiClient, locationRequest, this);
	}
}
\end{lstlisting}

\subsection{Route Matching Algorithm}\label{sec:routematchingalgorith}
The implementation of the route-matching algorithm is delegated to the previously mentioned outdoor location based service group, as they are responsible for implementation in the \gls{astep} system.
They were given the pseudocode algorithm developed in sprint 2 and cooperated with during the implementation process.
The implementation will not be documented here, as the actual implementation is not performed by us.
However, the implementation should be tested, and we generated test data with accompanying result expectations.
The test data was also handed over, this will be described in the following test section.

In cooperation with the outdoor group, it was decided to send every variable for the algorithm as parameters in the API call.
This was done so we would have control over the algorithm to ensure it performed as we saw fit.