\section{Implementation}

\subsection{Location Gathering Background service}
% metatext
This subsection will describe the contents of the implementation of the background location gathering.
The new implementation is based on the poor results \todo{can we document the poor results?} of the location gathering method implemented in sprint 2 that was not working as intended.
The code examples in this section will not include potential logging and debugging lines as to save space and they do not influence the functionality of the code.


% activityRecognitionInstantiation
The location gathering should first detect that the device is in a driving activity and then begin recording locations, according to the design described in Section \ref{ssec:LocationGatherBGS}.
To achieve the functionality, the app must connect to a Google API client and subscribe to activity recognition.
The code for subscribing to the activity recognition is shown in Listing \ref{activityRecognitionInstantiation}.
The activity recognition can thereafter check that the detected activity is a driving activity, and accordingly activate the location gathering.

\begin{lstlisting}[language=Java, label=activityRecognitionInstantiation, caption=Initialization of activity recognition.]
if(activityDetectionBroadcastReceiver == null) {
	activityDetectionBroadcastReceiver = 
		new ActivityDetectionBroadcastReceiver(googleApiClient, this);
		
	LocalBroadcastManager.getInstance(this).registerReceiver(
		activityDetectionBroadcastReceiver, 
		new IntentFilter("fapptory_inc.rideshare.BROADCAST_ACTION"));
}
\end{lstlisting}

The \texttt{activityDetectionBroadcastReceiver} is an instantiation of the\\ \texttt{ActivityDetectionBroadcastReceiver} class from, utilizing the \texttt{googleApiClient} and \texttt{this} which in this case references the current \texttt{MainActivity} instance to broadcast detected activities.

The broadcast receiver is then applied by registering to the \gls{rs} app's\\ \texttt{BROADCAST\_ACTION}, so that the \texttt{activityDetectionBroadcastReceiver} only handles broadcasts sent by \gls{rs} and not by other apps.

% ActivityIntentService broadcast description



% Activity Detection Broadcast Receiver
The main operations in regards to activity detection and location gathering are done by the activity detection broadcast receiver.

When the \texttt{activityDetectionBroadcastReceiver} receives a RideShare broadcast, the extra data in the broadcast is stored in the \texttt{ArrayList<DetectedActivity> detectedActivities}.
The \texttt{detectedActivities} is iterated over, shown in Listing \ref{activityRecognitionIteration} to find the driving activity \texttt{DetectedActivity.IN\_VEHICLE}.
The \texttt{startLocationUpdates()}, which is described later, is called if the confidence is above the threshold and if locations are not currently being recorded.
Line 8 controls the location cache in case a drive is stopped temporarily and starts again, appending the cached route to the actual route.

\begin{lstlisting}[language=Java, label=activityRecognitionIteration, caption=Iteration over received list of activity recognition.]
for (DetectedActivity da: detectedActivities){
	if(da.getType() == DetectedActivity.IN_VEHICLE){
		if(da.getConfidence() >= ACTIVITY_CONFIDENCE_VALUE){
			if(!isCurrentlyCollectingLocations) {
				startLocationUpdates();
				isCurrentlyCollectingLocations = true;
			}
			if(potentialStopDrivingActivity && potentialAstepRoute.size() > 0){
				astepRoute.addAll(astepRoute.size(), potentialAstepRoute);
				potentialAstepRoute.clear();
			}
			lastDetectedVehicleActivity = System.currentTimeMillis();
			potentialStopDrivingActivity = false;
			drivingActivityDetected = true;
		}
	}
}
\end{lstlisting}

% Location Updates Request
The \texttt{startLocationUpdates()}, shown in Listing \ref{startLocationUpdates}, is called to gather locations regularly.
The location updates request is based on the Google API client instantiated in \texttt{MainActivity} and a \texttt{locationRequest}.
The \texttt{locationRequest} is defined with properties regarding location update interval, fastest interval, and location quality priority, respectively set to \todo{agree upon constants} 60 seconds, 120 seconds and \texttt{PRIORITY\_HIGH\_ACCURACY}.

\begin{lstlisting}[language=Java, label=startLocationUpdates, caption=Start location updates functions.]
public void startLocationUpdates(){
	if(/* Omitted: Check permissions */) {
		LocationServices.FusedLocationApi
			.requestLocationUpdates(googleApiClient, locationRequest, this);
	}
}
\end{lstlisting}

To send the routes gathered through the location service, the communication with \gls{astep} must be established.

\subsection{App-to-aSTEP communication}
% ja..