\section{Implementation}
The implementation section contains descriptions of selected parts of the implementation performed in sprint 3.

The main focus is the location gathering service, which will record locations continuously while the user is driving, but not otherwise. 

\subsection{Location Gathering Background service}
% metatext
This subsection will describe the contents of the implementation of the background location gathering.
The new implementation is founded on the poor results of the location gathering method implemented in sprint 2, Section \ref{section:locationgathering}, that was not working as intended.
The code examples in this section will not include potential logging and debugging lines as for better readability and they do not influence the functionality of the code.

\textbf{Implementation plan}\\
According to the design, Section \ref{section:s3locgath}, the locations should first be requested when the user is driving.
Therefore, the implementation starts by instantiating an activity detector.
The activity detector can thereafter check the received activity, and begin location gathering if the activity is above a confidence threshold for driving activity.
When the confidence is lower than the threshold for a period of time, the driving activity can be considered finished, and the route must be stored and later transmitted to the \gls{astep} system.


% activityRecognitionInstantiation
\textbf{Activity recognition instantiation}\\
The location gathering should first detect that the device is in a driving activity and then begin recording locations, according to the design described in Section \ref{ssec:LocationGatherBGS}.
To achieve the functionality, the app must connect to a Google API client and subscribe to activity recognition.
The code for subscribing to the activity recognition is shown in Listing \ref{activityRecognitionInstantiation}.
The activity recognition can thereafter check that the detected activity is a driving activity, and accordingly activate the location gathering.

\begin{lstlisting}[language=Java, label=activityRecognitionInstantiation, caption=Initialization of activity recognition.]
if(activityDetectionBroadcastReceiver == null) {
	activityDetectionBroadcastReceiver = 
		new ActivityDetectionBroadcastReceiver(googleApiClient, this);
		
	LocalBroadcastManager.getInstance(this).registerReceiver(
		activityDetectionBroadcastReceiver, 
		new IntentFilter("fapptory_inc.rideshare.BROADCAST_ACTION"));
}
\end{lstlisting}

The \texttt{activityDetectionBroadcastReceiver} is an instantiation of the\\ \texttt{ActivityDetectionBroadcastReceiver} class from, utilizing the \texttt{googleApiClient} and \texttt{this} which in this case references the current \texttt{MainActivity} instance to broadcast detected activities.

The broadcast receiver is then applied by registering to the \gls{rs} app's\\ \texttt{BROADCAST\_ACTION}, so that the \texttt{activityDetectionBroadcastReceiver} only handles broadcasts sent by \gls{rs} and not by other apps.

% ActivityIntentService broadcast description



% Activity Detection Broadcast Receiver
\textbf{Activity Detection Broadcast Receiver}\\
The main operations in regards to activity detection and location gathering are done by the activity detection broadcast receiver.

When the \texttt{activityDetectionBroadcastReceiver} receives a RideShare broadcast, the extra data in the broadcast is stored in the \texttt{ArrayList<DetectedActivity> detectedActivities}.
The \texttt{detectedActivities} is iterated over, shown in Listing \ref{activityRecognitionIteration} to find the driving activity \texttt{DetectedActivity.IN\_VEHICLE}.
The \texttt{startLocationUpdates()}, which is described later, is called if the confidence is above the threshold and if locations are not currently being recorded.
Line 8 controls the location cache in case a drive is stopped temporarily and starts again, appending the cached route to the actual route.

\begin{lstlisting}[language=Java, label=activityRecognitionIteration, caption=Iteration over received list of activity recognition.]
for (DetectedActivity da: detectedActivities){
	if(da.getType() == DetectedActivity.IN_VEHICLE){
		if(da.getConfidence() >= ACTIVITY_CONFIDENCE_VALUE){
			if(!isCurrentlyCollectingLocations) {
				startLocationUpdates();
				isCurrentlyCollectingLocations = true;
			}
			if(potentialStopDrivingActivity && potentialAstepRoute.size() > 0){
				astepRoute.addAll(astepRoute.size(), potentialAstepRoute);
				potentialAstepRoute.clear();
			}
			lastDetectedVehicleActivity = System.currentTimeMillis();
			potentialStopDrivingActivity = false;
			drivingActivityDetected = true;
		}
	}
}
\end{lstlisting}

% Location Updates Request
\textbf{Location Updates Request}\\
The \texttt{startLocationUpdates()}, shown in Listing \ref{startLocationUpdates}, is called to gather locations regularly.
The location updates request is based on the Google API client instantiated in \texttt{MainActivity} and a \texttt{locationRequest}.
The \texttt{locationRequest} is defined with properties regarding location update interval, fastest interval, and location accuracy priority, respectively set to 60 seconds, 60 seconds and \texttt{PRIORITY\_HIGH\_ACCURACY}.

\begin{lstlisting}[language=Java, label=startLocationUpdates, caption=Start location updates functions.]
public void startLocationUpdates(){
	if(/* Omitted: Check permissions */) {
		LocationServices.FusedLocationApi
			.requestLocationUpdates(googleApiClient, locationRequest, this);
	}
}
\end{lstlisting}

\subsection{Route Matching Algorithm}
The implementation of the route matching algorithm is delegated to the previously mentioned outdoor location based service group, as they are responsible for implementation in the \gls{astep} system.
They were given the pseudocode algorithm developed in sprint 2 and cooperated with during the implementation process.
The implementation will not be documented here, as the implementation is not performed by us.
However, the implementation should be tested, and we generated test data with accompanying result expectations.
The test data was also handed over, but this will be described in the following test section.



\subsection{App-to-aSTEP communication}
%meta text
An overview of the implementation of the API calls to \gls{astep} will be described in the following chapter.

In previously design phases it were decided what API calls should be implemented in the app.
On \ref{tab:asteprequests} a specified list of calls only to be performed on the app can be seen.

\begin{table}[h]
	\centering
	\scriptsize
	\begin{tabular}{l l l}
		Path & Method & Description\\\midrule
		/locations/outdoor/\{username\}/routes & POST & Send a route to the \gls{astep} server\\
		/users & POST & Create a new user\\
		/users/\{username\}/token & GET & Get the current valid token for a user\\
		/users & POST & Invalidate old token and get new\\
		/users/\{username\}/token & POST & Invalidates previous token for a user\\
	\end{tabular}
	\label{tab:asteprequests}
	\caption{}
\end{table} 

The API calls are implemented using a \texttt{HttpURLConnection} class, which is a \texttt{URLConnection} class for HTTP specific data transfer over the web.
 \texttt{HttpURLCOnnection} may be used to send and receive streaming data whose length is not known in advance.
 This is needed as every API call varies in size.
 Looking at the API call to post a route, it will be the one that varies in size the most as every routes will differ in size.



Using a object orientate structure, each API call is implemented to overwrite a \texttt{helperClass} which will handle the inputs, such as username, password, token, and routes.
The inputs is then transformed to a part of an URL which would fit each of the relevant API calls.

The helperClass is then parsed to the \texttt{HttpURLConnection} which will use a string builder to create the final URL from the API base URL, \texttt{http://astep.cs.aau.dk:8080/api/}.
Lastly the \texttt{HttpURLConnection} is trying to perform the API call, from where response codes are handled.

It should be noted that passwords are not in the URL part of the API call, as these will be logged on the \gls{astep} server. 
This would be a major security issue.
Instead are password and tokens parsed as part of the body, which is used by REST. 
This will hide the content from the URL and instead be transfered in an \texttt{OutputStream} encoded in raw bytes. 

On two examples of API calls can be seen on \ref{list:example}.
Post route is simplified to reduce size and make it easier to explain.

\begin{enumerate}
\label{list:example}
\item POST new user
\begin{enumerate}
\item URL: http://astep.cs.aau.dk:8080/api/users?username=USERNAME
\item CURL: curl -X POST --header 'Content-Type: application/json' --header 'Accept: application/json' -d 'PASSWORD' 'http://astep.cs.aau.dk:8080/api/users?username=USERNAME'
\end{enumerate}
\item POST route to \gls{astep}
\begin{enumerate}
\item URL: http://astep.cs.aau.dk:8080/api/locations/outdoor/outdoor/routes?authorization=TOKEN\&route=longitude,latitude,precision,timestamp;\&route=longitude,latitude,precision,timestamp
\&distance\_weight\&time\_weight\&largest\_acceptable\_detour\\
\_length=146\&acceptable\'httptime\_difference=32
\item CURL: curl -X POST --header 'Content-Type: application/json' --header 'Accept: application/json' 'URL from above'
\end{enumerate}
\end{enumerate}

As seen on Create user in \ref{list:example}, the username is parsed in the URL while the password is included in a header.
On the other hand is the post route only parsed in the URL as there are no information that needs to be hidden.
Every decision made around parsing the data from the app to the aSTEP server is decided by the individual aSTEP API groups.


%hey

%farvel

