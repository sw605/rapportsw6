A new background service is examined as the background service implemented and tested in Section \ref{sec:s2test} was not performing correctly.
Furthermore, the constant interval used in the previous background service between gathering GPS location is not ideal, as a user could be stationary for periods of time, thus the location gathering is performed unnecessarily when the user is e.g. sleeping.

\textbf{Location Gathering}\\
Since Section \ref{section:locationgathering} we have discovered a new method for gathering locations. 
It can be done by requesting location updates from the Google Play Services \cite{receivingLocationUpdates}.
By this method, an app can receive regular location updates by subscribing to a location provider.
The app could then receive a notification when a new location is detected, and rely on the Google implementation which is already optimized within Android.

\textbf{Activity Detection}\\
% energy preserving
The \gls{rs} app is supposed to gather locations when the user is driving, and hence activity recognition is explored to be able to detect the user's activity and potentially reduce energy consumption.

The activity detection could improve two factors: Energy consumption and location filtering.
The location filtering should be done by only recording locations when the user is driving.
The energy consumption is lowered as the GPS unit is only utilized when the locations are necessary to collect.
