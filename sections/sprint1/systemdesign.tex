% Divided into structual & behavioal design
\subsubsection{Structual Design}
The application requires much processing power to analyze all the possible combinations of routes.
Using a heuristic rule to filter the routes before analyzing them reduces this problem.
But, it is still helpful to handle the analysis away from the smartphone due to battery and privacy concerns.
So a lot of the analysis must take place on the central server.

\begin{figure}[h]
	\centering
	\begin{tikzpicture}[auto]
	\node[draw, minimum height=2em](Server){aSTEP Server};
	\node[draw, cylinder, shape border rotate=90, aspect=0.25, right=of Server](Data){Data};
	\node[draw, minimum height=2em, left=of Server](App){Application};
	
	\draw[->] (App) edge (Server) (Server) edge (Data);
\end{tikzpicture}
	\caption{The over all stucture.}
	\label{fig:packageDiagramSprint1}
\end{figure}

If we assume that this central server can handle the analysis of the routes, the application need only to supply the server with routes which to analyze.
A small process which runs in the background of the device does this.
As seen in Figure \ref{fig:classDiagramSprint1Observer} the Observer class' main goal is to observe what activity the user is currently doing.
If the activity is driving then it starts to store the locations with the RouteBuilder class.
When the Observer detects that the user stopped driving, the Observer sends the route to the server.

\begin{figure}[h]
	\centering
	\begin{tikzpicture}[auto]
	\node[class](Observer){Observer
		\nodepart[align=left, text width=4.1cm]{two}
		-currentActivity : Integer
		\nodepart[align=left, text width=4.1cm]{three}
		+observerActivity() : Integer};
	\node[class, below=of Observer](RouteBuilder){RouteBuilder
		\nodepart[align=left, text width=4.9cm]{two}
		-locations : Location[]\\
		-user : UserInfo
		\nodepart[align=left, text width=4.9cm]{three}
		+RouteBuilder(user : UserInfo)\\
		+addLocation(location : Location)\\
		+buildRoute() : Route};
	\draw[aggregation] (Observer) -- (RouteBuilder);
\end{tikzpicture}
	\caption{The small process which should run at all time to build the routes.}
	\label{fig:classDiagramSprint1Observer}
\end{figure}

On the server, the route is stored in the database and forwarded to the RouteStabilityAnalyser.
This analyzer takes in the route and compares it to other routes by the same user, to determine whether this route is a regular occurrence.

If the RouteStabilityAnalyser determines that the route is stable.
Then this information gets stored in the database together with when this route occurs.
The server then uses the RouteSimilarityAnalyser to find good matches for the new StableRoute.
The matches get stored and new matches are then sent to the device next time they have contact.

\begin{figure}[h]
	\centering
	\begin{tikzpicture}[auto]
	\node[class](Route){Route 
		\nodepart[align=left, text width=6.3cm]{two}
			-id : Integer\\
			-user : UserInfo\\
			-points : Location[]
		\nodepart[align=left, text width=6.3cm]{three}
			+Route(user : UserInfo, points : Location[])\\
			+getUser() : UserInfo\\
			+getPoint(index : Integer) : Location};
	\node[class, below=of Route](StableRoute){StableRoute
		\nodepart[align=left, text width=6.3cm]{two}
			-days : WeekDays\\
			-departureTime : Time\\
			-arrivalTime : Time
		\nodepart[align=left, text width=6.7cm]{three}
			+StableRoute(route : Route, days : WeekDays)\\
			+getArrivalTime() : Time\\
			+getDepartureTime() : Time};
	\node[class, below=of StableRoute](RouteStabilityAnalyser){RouteStabilityAnalyser
		\nodepart[align=left, text width=4.6cm]{two}
			-isStable : Boolean\\
			-currentRoute : Route
		\nodepart[align=left, text width=4.6cm]{three}
			+getStableRoute() : StableRoute\\
			+analyse(route : Route)};
	\node[class, right=of StableRoute](RouteAnalyser){\textit{RouteAnalyser}
		\nodepart[align=left, text width=3.4cm]{two}
			-id : Integer
		\nodepart[align=left, text width=3.4cm]{three}
			+analyse(route : Route)};
	\node[class, right=of RouteStabilityAnalyser](RouteSimilarityAnalyser){RouteSimilarityAnalyser
		\nodepart[align=left, text width=3.4cm]{three}
			+analyse(route : Route)};
	\node[interface, below=5 cm of RouteAnalyser](RouteComparer){
		\nodepart[align=center, text width=4cm]{one}
		<<interface>>\\
		IRouteComparer
		\nodepart[align=left, text width=5.9cm]{two}
			+compare(x : Route, y : Route) : Integer};
	\node[class, below=of RouteComparer](TimeDistanceComparer){TimeDistanceComparer
		\nodepart[align=left, text width=5.9cm]{three}
			+compare(x : Route, y : Route) : Integer};
	
	\draw[aggregation,transform canvas={xshift=1.3cm}] (RouteAnalyser) -- (RouteComparer);
	\draw[association, dashed,transform canvas={xshift=-1.3cm}] (RouteStabilityAnalyser) -- node[label]{<<build>>} (StableRoute);
	
	\draw[generalization] (StableRoute) -- (Route);
	\draw[generalization] (RouteStabilityAnalyser.north) -- ++(0,0.5) -| (RouteAnalyser);
	\draw[generalization] (RouteSimilarityAnalyser.north) -| (RouteAnalyser);
	\draw[generalization, dashed] (TimeDistanceComparer) -- node[label]{<<realize>>} (RouteComparer);
	\draw[association] (RouteAnalyser) |- node[label, at end, xshift=0.3cm]{0..*} (Route.east);
\end{tikzpicture}
	\caption{The structure of the system inside the server. This package should be an isolated system within the server.}
	\label{fig:classDiagramSprint1Server}
\end{figure}

The application acts as an interface to the background process for the user.
This is where the user logs in and can see an overview of candidates for car pooling.

\begin{figure}[h]
	\centering
	\begin{tikzpicture}[auto]
	\node[class](LoginActivity){LoginActivity
		\nodepart[align=left, text width=2.7cm]{two}
			-username : String\\
			-password : String
		\nodepart[align=left, text width=2.7cm]{three}
			+submit()};
	\node[class, below left=1 and -1 of LoginActivity](RegisterActivity){RegisterActivity
		\nodepart[align=left, text width=2.7cm]{two}
			-username : String\\
			-password : String
		\nodepart[align=left, text width=2.7cm]{three}
			+submit()};
	\node[class, below right=1 and -1 of LoginActivity](MainActivity){MainActivity
		\nodepart[align=left, text width=4cm]{two}
			-currentUser : UserInfo
		\nodepart[align=left, text width=4cm]{three}
			+MainActivity(user : UserInfo)\\
			+fetchSuggestedRoutes()};
	\node[interface, right=of MainActivity](ISuggestedRoute){
		\nodepart[align=center, text width=2.8cm]{one}
			<<interface>>\\
			ISuggestedRoute
		\nodepart[align=left, text width=2.8cm]{two}
			-getRoute() : Route};
	\node[class, below=of ISuggestedRoute](SuggestedRoute){SuggestedRoute
		\nodepart[align=left, text width=2.8cm]{two}
			-getRoute() : Route};
		
	\draw[association, dashed] (LoginActivity) -| (MainActivity);
	\draw[association, dashed] (LoginActivity) -| (RegisterActivity);
	\draw[association, dashed] (RegisterActivity) -- (MainActivity);
	\draw[generalization, dashed] (SuggestedRoute) -- node[label] {<<realize>>} (ISuggestedRoute);
	\draw[aggregation](MainActivity)--(ISuggestedRoute);
	
	\node[draw, fit=(MainActivity) (LoginActivity) (RegisterActivity)](PackageBody){};
	\node[draw, above right] at(PackageBody.north west){activities};
\end{tikzpicture}
	\caption{The structure of the application it self.}
	\label{fig:classDiagramSprint1Application}
\end{figure}

This is the start of a project that will be further developed and expanded in the future.
So reuse is important to reduce unused server resources.
By isolating parts which were specific to the rideshare app, most of the system would be reusable.
The best example of this is the RouteAnalyser from Figure \ref{fig:classDiagramSprint1Server}.
This analyzer filter routes according to its concrete implementation.
After the filter, the RouteAnalyser uses the strategy pattern to determine how to compare the routes.
This enables other projects to develop their own filters and comparers in later projects.

To reduce the search space, a weekly stability was decided. 
This means that the RouteStabilityAnalyser can limit itself to analysis a month or two of data.
Later on this needs to be tested and evaluated upon.