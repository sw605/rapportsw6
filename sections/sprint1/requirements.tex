\section{Requirement Specification}\label{sec:req}
% Metatext
The requirements are divided into two main categories, functional and nonfunctional requirements, and are furthermore sorted in the MoSCoW \cite{moscow} structure.
The structure assists in solving the core purpose of the system first, as these are the highest prioritized requirements, and later adding additional functionality.


% Requirements
\subsection{Functional requirements}
The functional requirements are directly related to the tasks and operations of the developed application.

\textbf{Must-have requirements}\\
The 'Must have' requirements must be fulfilled for the solution to be acceptable.

\textit{Graphical user interface}\\
The application must have a graphical user interface (GUI) as users must be able to use the application themselves, and to structurally and aesthetically display user matches. 
The GUI must be intuitive as users may have different levels of experience with using mobile applications.

\textit{User accounts, including login and registration.}\\
Unique user accounts are required as it serves as an identifier for each user in the system. 
With user accounts, store personal information, and compare user based on the data. 
It will also provide functionality for displaying user to other users.

\textit{Communication with the \gls{astep} system.}\\
As the project is a part of the bigger system \gls{astep}, there must be a relation to the developed platform. 
The communication would be storing data in the \gls{astep} database and using user management also implemented on the platform.

\textit{User location tracking and storage.}\\
The application must track users as they move, to be able to determine the users travel routes.
The application should store the commuted routes and the data should be stored in the \gls{astep} database.

\textit{Automatically determine regular routes.}\\
The solution must automatically determine if a route is regular. 
Regular routes should be used for route comparison and match generating. 

\textit{Automatic user ridesharing suggestions.}\\
Automatic matching with other users routes must be provided to the the individual app users, and the matches must be computed by the solution.
When two regular routes are found similar, the users must receive an indication of the match in the app.
If the match computation shows to be a excessive process it is deemed to be done in the \gls{astep} core.


\textbf{Should-have requirements}\\
The requirements in this subsection are important, but are not regarded as critical for the functionality of the solution.

\textit{Give the user option to specify whether they have a car or not.}\\
Because some users might not have a car, they should be considered differently than the ones with, as the users without vehicle cannot give a ride to a potential match.

\textit{Enable users to blacklist other users.}\\
Users should be able to filter other users from the matches to prevent further possible bad experiences with either a driver or passenger.

\textit{Suggest rides with users who only drives a subset of the way from A to B.}\\
Giving users the option to match with people who do not have the same source and destination, will increase the number of suggested matches, as a user can join a part of the matches' route.


\textbf{Could-have requirements}\\
The 'Could have' requirements are the lowest realistically fulfillable requirements, and are implemented if the higher priority requirements are fulfilled.

\textit{Ride reservation or request from, to, time.}\\
This requirement enables the user to reserve rides with other users. 

\textbf{Would-have requirements}\\
The following requirements are only considered when all other requirements are satisfied but initially regarded as tasks to be solved in future projects.

\textit{Inform users of their environmental and economic savings due to their use of the solution.}\\
Provide users with detailed information on how much fuel the user has saved, how much money saved based on fuel prices, and how much CO2 that is not released into the environment. 

\subsection{Non-functional requirements}
The nonfunctional requirements for the solution are stated in the following paragraphs.

\textbf{Must-have requirements}\\
These are the requirements the solution must fulfill to be acceptable.

\textit{Development cooperation with the other \gls{astep} project groups.}\\
The development must be done in cooperation with the other \gls{astep} groups.

\textit{User privacy}\\
The app must respect user privacy, especially in regards to a user location data and personal user data.
The solution should consider securing elements such as database storage, user management, and developers.

\textbf{Should-have requirements}\\
These requirements are important, but are not regarded as highly critical.

\textit{Aesthetics matching other \gls{astep} project applications}\\
The app should share design and guidelines as the other apps developed for \gls{astep}.

The first design phase can be initiated as the solution requirements now are specified. 