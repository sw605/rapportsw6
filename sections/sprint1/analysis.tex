\section{Analysis}
Ridesharing is an activity which have had different time of popularity in recent history according to Nelson D. Chan \& Susan A. Shaheen\cite{doi:10.1080/01441647.2011.621557}.
The latest rise of popularity in ridesharing they attribute to the organizational capabilities of technologies such as the Internet and smartphones.
They also list insentiences for this phase of ridesharing which are as follows 
\begin{itemize}
  \item ``Focus on reducing climate change, growing dependence on foreign oil, and traffic congestion
  \item Partnerships between ridematching software companies and regions and large employers
  \item Financial incentives for green trips through sponsors
  \item Social networking platforms that target youth
  \item Real-time ridesharing services'' according to \cite{doi:10.1080/01441647.2011.621557}.
\end{itemize}

This presents a broad overview of why ride sharing became popular again around 2014.
The following section documents the analysis of both academic papers, as well as corporate and community products, to get a comprehension of the state of the art in the ridesharing field.
In the analysis, the major focus will be on what is called real-time or dynamic ridesharing which Andrew Amey et al. defined as ``rideshare service relying heavily on mobile phone technologies''\cite{amey2011real}.

\subsection{Scholar Papers}
The problem of automating ridesharing through modern technology has already been studied by multiple scholars\todo{what?} with different approaches. The following sections briefly account for some of which we found the most interesting and relevant to the problem statement. 

Amey et al., and Chan \& Shaheen sees technology as one of the most important factors in the present and future of dynamic ridesharing \cite{doi:10.1080/01441647.2011.621557, amey2011real}.


In ``T-Share: A Large-Scale Dynamic Taxi Ridesharing Service'' \cite{ShuoMa2013} a solution for taxis picking up multiple passengers throughout a ride is developed and presented, seeking to increase the passenger throughput as well as lowering the distance traveled per passenger served.
Specifically interesting in aforementioned paper is an approach which suggest a grid representation of a map of a road network, this is to avoid shortest path calculations between points on a map, which is expensive, and instead approximate distance represented by cells in the grid.
The paper also, as well as most\todo{add source} other services, utilize the user's smartphone to collect location data and act as a user interface.

``Real-time rideshare matching problem'' researches what might matter when matching driver and passengers.
Their uses focuses on preferences which influences whether you want drive with a person, this can be everything from gender to pet friendliness.
They developed an algorithm which assess whether a passenger and driver match, based on location, preferences, passenger and/or driver detour as the most important factors \cite{ghoseiri2011real}.
This algorithm might be useful in this project solution.

Actual choices and influences regarding algorithm design for ridematching will be addressed in the design phase. 

\subsection{Private Solutions}
Concerning corporate and community solutions which exist these are quite different form the academic field.
Overall, the main focus seems to be centered around taxi alternatives.
The two biggest international players\todo{services?} in this field is Uber and Lyft\todo{source}.
There are also several services with some or all of their core focus in Denmark, most notable is services like Haxi and Drivr\todo{source}.
These services works in a similar way: 
\begin{enumerate}
	\item A customer request a ride through an smartphone app (or maybe a web interface)
	\item The backend of the service sends the request to one or more appropriate drivers
\end{enumerate}
Since the mentioned services are commercial and closed-source, actual information about the services design or architecture is not available.
Drivr has two interesting feature in comparison to their opponents, that is a web system for respectively fleet management and business which provides administration and expense control of employee taxi travels.

In addition to these taxi like services there also exist services which focuses on traditional ridesharing between private people where money is not earned, but expenses may be shared.
Again, here exist both services that tries to reach internationally and services that are focusing on certain areas or countries.
These services are straightforward\todo{simple in regards to something?} and offer the opportunity to either offer a lift, request a ride and the possibility to connect drivers and passengers.
These are usually free, however some of the services only charge a small fee when connecting people.
Some of the more notable are GoMore and iRideshare.\todo{source}

The smartphones have automatized much of the labor concerning organized ridesharing and is acts as a central component in all of the analyzed systems\todo{what?}.
However, we still see an opportunity for the smarthpone to be utilized even further in a service arranging ridesharing, thus decreasing the required labor and potentially increase the popularity of ridesharing.