\section{Analysis}
Ridesharing is an activity which have had different time of popularity in recent history according to Nelson D. Chan and Susan A. Shaheen\cite{doi:10.1080/01441647.2011.621557}.
The latests rise of popularity in ridesharing they attribute to the organizational capabilities of technologies such as the Internet and smartphones.
They also list insentiences for this phase of ridesharing which are as follows 
\begin{itemize}
  \item ``Focus on reducing climate change, growing dependence on foreign oil, and traffic congestion
  \item Partnerships between ridematching software companies and regions and large employers
  \item Financial incentives for green trips through sponsors
  \item Social networking platforms that target youth
  \item Real-time ridesharing services'' according to \cite{doi:10.1080/01441647.2011.621557}.
\end{itemize}

This present a broad overview of why ride sharing became popular again around 2014.
In the following section we analyze both academic papers as well as corporate and community products, to get a comprehension of the state of the art in the ridesharing field.

\subsection{Scholar Papers}
The problem of automating ridesharing though modern technology has already been studies by multiple scholars with different approaches, herein we briefly account for some of which we found most interesting and overlapping with our problem. 

In ``T-Share: A Large-Scale Dynamic Taxi Ridesharing Service'' \cite{ShuoMa2013} a solution for taxis picking up multiple passengers throughout a ride is developed and presented, seeking to increase the passenger throughput as well as lowering the distance traveled per passenger served.
Specifically interesting in aforementioned paper is an approach which suggest a grid representation of a map of a road network, this is to avoid shortest path calculations between points on a map, which is expensive, and instead approximate distance represented by cells in the grid.
The paper also, as well as most other services, utilize the users smartphone to collect location data and act as user interface.


\subsection{Private Solutions}
Looking on corporate and community solutions the market are quite different form the academic field. 
More to come here...