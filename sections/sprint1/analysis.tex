\section{Analysis}
Ridesharing is an activity which have had different time of popularity in recent history according to Nelson D. Chan and Susan A. Shaheen\cite{doi:10.1080/01441647.2011.621557}.
The latests rise of popularity in ridesharing they attribute to the organisational capabilities of technologies such as the Internet and smartphones.
They also list insentiences for this phase of ridesharing which are as follows 
\begin{itemize}
  \item ``Focus on reducing climate change, growing dependence on foreign oil, and traffic congestion
  \item Partnerships between ridematching software companies and regions and large employers
  \item Finacial incentives for green trips through sponsors
  \item Social networking platforms that target youth
  \item Real-time ridesharing services'' according to \cite{doi:10.1080/01441647.2011.621557}.
\end{itemize}


The problem of automating ridesharing has already been studies by multiple scholars with different approaches in this section we briefly account for some of which we found most interesting and overlapping with our problem. 

An interesting approach is taken ``T-Share: A Large-Scale Dynamic Taxi Ridesharing Service'' where a solution for taxis picking up multiple passengers throughout a ride, seeking to increase the passenger throughput as well as lowering the distance traveled per passenger served\cite{ShuoMa2013}.
Specifically interesting for us the aforementioned paper suggest a grid representation of a map of a road network, to avoid expensive shortest path calculations between points on a map.