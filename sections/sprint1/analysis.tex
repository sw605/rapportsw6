\section{Analysis}
In this section some basics analysis will be performed in regards to formally defining ridesharing, study papers and commercial solutions. 

\subsection{Defining Ridesharing}
Ridesharing is an activity which has had phases of popularity in recent history according to \citet{doi:10.1080/01441647.2011.621557}.
They describe the current phase as the technology-enabled ridematching phase which started in 2004.
In this phase, the integration of the internet, mobile phones, and social media into ridesharing services reduces the barrier to entry for new potential passengers and drivers.
\citet{doi:10.1080/01441647.2011.621557} also list incentives for ridesharing in the current phase, which are the following 

\begin{itemize}
  \item ``Focus on reducing climate change, growing dependence on foreign oil, and traffic congestion
  \item Partnerships between ridematching software companies and regions and large employers
  \item Financial incentives for green trips through sponsors
  \item Social networking platforms that target youth
  \item Real-time ridesharing services'' \citep{doi:10.1080/01441647.2011.621557}.
\end{itemize}

This presents a broad overview of why ridesharing became popular again after 2004.
Because technologies play a vital role in the current phase, a selection of popular commercial services will be examined a long with some scientific papers to get a comprehension of the state of the art in the field.
%The following section documents the analysis of both academic papers, as well as corporate and community products, to get a comprehension of the state of the art in the ridesharing field.
The main focus of this project will be on real-time and dynamic ridesharing which \citet{amey2011real} defined as ``rideshare service relying heavily on mobile phone technologies''.

\subsection{Scientific Papers}
The problems of automating ridesharing through modern technology has already been studied with different approaches.
The following sections account for some of which the project group found the most interesting and relevant to the problem statement. 

\citet{doi:10.1080/01441647.2011.621557, amey2011real} sees technology as one of the most important factors in the present and future of dynamic ridesharing.

\citet{ShuoMa2013} developed an algorithm for taxi services and improved throughput of passengers by 25\% and reduced the distance a single taxi had to drive by 13\%, at six requests for rides per taxi.
 %  picking up many passengers throughout a ride is developed and presented, seeking to increase the passenger throughput as well as lowering the distance traveled per passenger served.
Especially interesting in their paper, is the approach of a grid representation of a road network, to avoid shortest path calculations between points on a map, and instead approximate distance based on cells in the grid.
The paper states that the solution also utilizes the user's smartphone to collect location data and act as a user interface.

\citet{ghoseiri2011real} researched what properties might matter when matching driver and passengers.
They focused on preferences which influence whether you want to drive with a person or not. The preferences can be properties such as gender or pet friendliness.
They developed functions that can assess if a passenger and driver match, based on location, preferences, passenger and/or driver detour as the most important factors \cite{ghoseiri2011real}.
This algorithm might be useful in this project solution.

Actual choices and influences regarding algorithm design for ridematching will be addressed in the design phase in Section \ref{sprint1design}.

\subsection{Commercial Solutions}
Corporate and community solutions are quite different from the academic field solutions.
The main focus seems to be centered more around taxi alternatives.
The are plenty of services around the world, with the two biggest international services in this field as of 2015 seems to be Uber and Lyft\cite{ridehail}.
There are also several services with some or all of their focus in Denmark, most notable is services like Haxi and Drivr\ref{drivrS}.
These services work mostly in a similar way: 

\begin{enumerate}
	\item A customer request a ride through an smartphone app or a web interface
	\item The backend of the service sends the request to one or more appropriate drivers
	\item A driver accepts the request, and dispatches
\end{enumerate}

Since the mentioned services are commercial and closed-source, actual information about the service's design or architecture is not available.

Drivr offers two interesting feature in comparison to their opponents, that is a web interface for fleet management and business which provides administration and expense control of employee taxi travels.

Besides these taxi-like services, there also exist services that focus on traditional ridesharing between private people, where money is not earned.
Here exist both local and international services. 
These services give the opportunity to either offer a lift, request a ride and the possibility to connect drivers and passengers.
These services are usually free, but some of them charge a small fee when connecting people.
Most notable is the service GoMore.

The smartphones have automatized much of the labor concerning organizing ridesharing and act as a central component in the analyzed systems.
However, we still see an opportunity to utilize the smartphone even further in a service that arranges rides between users, thus decreasing the actual required user actions.\todo{add arguments}
