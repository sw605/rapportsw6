\pdfbookmark[0]{Title page}{label:titlepage_en}
\aautitlepage{%
  \englishprojectinfo{
    Project Title %title
  }{%
    Scientific Theme %theme
  }{%
    Spring Semester 2016 %project period
  }{%
    SW605F16 % project group
  }{%
    %list of group members
	Mathias C. Mikkelsen\\
    Bjørn E. Opstad\\ 
	Morten Pedersen\\
    Claus W. Wiingreen
  }{%
    %list of supervisors
    David Frazetto
  }{%
    1 % number of printed copies
  }{%
    \today % date of completion
  }%
}{%department and address
  \textbf{Department of Computer Science}\\
  Aalborg University\\
  \href{http://www.cs.aau.dk}{http://www.cs.aau.dk}
}{% the abstract
  The semester project was to develop an app that utilizes locations in collaboration with the \acrlong{astep} system, which also needed to be developed.
  The \acrlong{rs} solution is a system that can gather user-driven routes and match with other users traveling similar routes so that the users could share vehicles.
  The solution could be advantageous in multiple aspects, by reducing the number of vehicles on the road, hence reducing traffic, jams, and emission.
  The solution consists of an Android app and a solution server and utilizes the \acrlong{astep} system to store gathered data, analyze user routes and suggest rideshare partners to the app users.
  The solution is in an acceptable state but needs a user interface refinement and a large-scale test to confirm the matching algorithm is working as intended for a bigger user basis.
}