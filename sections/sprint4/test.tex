\section{Test}
% metatext for test section
This section will contain descriptions and documentation of the tests performed on the whole \gls{rs} solution in the fourth sprint.
The tests are the final tests on the solution, and will serve as a full system test and as the basis for the solution and project evaluation.
The tests are distributed into categories of the different parts in the solution: vertices(things) and edges(communication).
Even though not all elements described in this section are described in the previous implementation sections, they are considered, as they are part of the solution.

Because of the short remaining project time, the tests are performed informally. 
However, the tests yield practical implementation results, and reflect the functionality of the solution.


\subsection{RideShare App}
\textbf{Background location gathering}\\
The background location caching test was performed in sprint 3, and was evaluated being fully functional, and will not be tested further here.

\textbf{Display matches to user}\\
\todo{fill in}


\subsection{aSTEP}
Our contribution to the \gls{astep} system was the pseudo code for the algorithm.
The algorithm was tested in sprint 3 and evaluated as sufficient for the solution, although missing a more comprehensive test with actual user routes.


\subsection{RideShare Server}
\textbf{some category tested}\\
works quite well, preferably.

\textbf{another tested category}\\
assumed working!


\subsection{App - aSTEP Communication}
\textbf{Route push}\\
Pushing routes with around 10 values is working correctly, as presented in the location gathering test in Section \ref{subsec:bgstest2}.
The \gls{astep} system is, however, limited by a 2000\todo{check} character limit for the URL length.
This limits the length of the supported routes to push to \gls{astep} to around 60\todo{check} locations in a route.

\textbf{Register new user}\\
Registering a user with English letters is tested and performs as expected.

\textbf{Login}\\
Calling the \gls{astep} API with the correct arguments yield an ok signal and a token.
The token is successfully utilized to authorize other API calls.

However, there are problems logging in with passwords and usernames containing Scandinavian letters, as can be seen in Table \ref{tab:logintest}.

This seems to be a problem with Android not incorporating the mentioned letters as standard, because the \gls{astep} website API responds positively for the tested values.
This is a usability problem with the \gls{rs} app, and could be solved by implementing the support of international letters.

\begin{table}[!ht]
	\centering
	\begin{tabular}{@{}lll@{}}
		Username & Password & Result \\
		\hline
		a & a & 200 OK\\
		ø & ø & 404 not found\\
		å & a & 404 not found\\
		b & å & 401 unauthorized\\
	\end{tabular}
	\caption{Registered users login test.}
	\label{tab:logintest}
\end{table}


\subsection{RideShare Server - aSTEP Communication}
The action performed between the \gls{rs} server and \gls{astep} is to receive route matches.
It works as intended.


\subsection{App - RideShare Server Communication}
\textbf{Get matches}\\
The \gls{rs} app receives matches goodly.

\textbf{Store extra data to new user}\\
Is working in beta as of May 16th.