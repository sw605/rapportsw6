% Connections
The server is implemented as a Java application running on a windows machine.
It communicates with the \gls{astep} API by using the \texttt{HttpURLConnection} class.
What is sent through these connections will be explained in later sections.

The app communicates with the server though the \texttt{ServerSocket} class and the \texttt{Socket} class.
These classes provide input and output streams which can be written to and read from.
A \texttt{BufferedStreamReader} reads the input stream as this class provides an easy to use interface for reading the input line by line.
The server writes to the output stream using the \texttt{PrintWriter} class.
This class is similar to the standard output stream in the way that it prints the string provided to the stream.

The messages send back and forth are of two types: Strings and JSON objects.
The simple strings are used for error messages such as \enquote{Unknown message} and \enquote{Invalid token} while the JSON objects are used for more complex results such as arrays of matches.

% Client requests
These following two messages are used on the clients.:

{\centering
	\begin{tabular}{l l}
		Update matches & \enquote{\texttt{UPD:[TOKEN]}}\\
		Register & \enquote{\texttt{REG:[TOKEN]:[PHONENUMBER]:[E-MAIL]:[FULL NAME]}}
	\end{tabular}
}

The server checks the received token sent through the \gls{astep} API and gets the username in both cases.
Update matches gets the updated set of matches for a user and register adds the user to the user details table in the database on the server.

%Main loop
The communication is contained in a loop that is implemented on the server, so that the server is ready for new clients as often as possible.
The loop is configured as seen in the following list:

\begin{enumerate}
	\item Load server configurations
	\item Open Server side socket
	\item If it is less than 24 hours since last match update from \gls{astep} goto 5
	\item Create a thread to get match updates from \gls{astep}
	\item Wait for a client
	\item Create a thread to handle client request in another thread
	\item goto 3
\end{enumerate}

The 24-hour interval, in line 3, was chosen to reduce the load on the \gls{astep} server, and minimizes work while presenting users with relevant information.
When receiving matches from \gls{astep}, the system runs the algorithms as described in Section \ref{ssec:algorithm} on routes that were received within the last two week.
The matches are then stored in the servers own database.

%Database
All contact to the database is handled in the Database class.
Because the class needs to load the database configurations from a file on the harddisk, it implements  the singleton pattern so that the file only have to be accessed once.
It would be inefficient if the server had to read this file every time it needs to access the database.

The database is only containing two tables: user details and matches.
The user details table keeps additional information about the users of the system, such as contact information and an alias they can use apart from their \gls{astep} username, stored as 'fullname'.
The table description can be seen in Table \ref{tab:userdetails}.

\begin{table}[h]
	\centering
	\begin{tabular}{l l l}
		Column 		& Type                   & Modifiers\\\hline
		username    & character varying(50)  & primary key\\
		phonenumber & character varying(30)  & not null\\
		email       & character varying(255) &\\
		fullname    & character varying(255) & not null\\
	\end{tabular}
	\caption{The table definition of user details in the \gls{rs} system.}
	\label{tab:userdetails}
\end{table}

A table of metadata for matches is stored on the \gls{rs} server, and contains the information that is specified in the system design. 
The table description can be seen in Table \ref{tab:matches}, where the different data types and names are defined.
The SQL statements used for creating the database can be found in Appendix \ref{appendix:database}.

\begin{table}[h]
	\centering
	\begin{tabular}{l l l}
			   Column  &            Type             & Modifiers\\\hline
			 user1     & character varying(50)       & references userdetails(username)\\
			 user2     & character varying(50)       & references userdetails(username)\\
			 starttime & timestamp without time zone & not null\\
			 score     & integer                     &
	\end{tabular}
	\caption{The table definition of matches in the \gls{rs} system.}
	\label{tab:matches}
\end{table}

