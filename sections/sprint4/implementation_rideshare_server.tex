% Connections
The server is implemented as a Java application running on a computer.
It communicates with the \gls{astep} API by using the \texttt{HttpURLConnection} class.
What is sent through these connections will be explained in later sections.

The app communicates with the server though the \texttt{ServerSocket} class and the \texttt{Socket} class.
These classes provide input and output streams which can be written to and read from.
A \texttt{BufferedStreamReader} reads the input stream as this class provides an easy to use interface for reading the input line by line.
The server writes to the output stream using the \texttt{PrintWriter} class.
This class is similar to the standard output stream in the way that it prints the string provided to the stream.

The messages send back and forth have two types: simple strings and JSON objects.
The simple strings are used for error messages such as \enquote{Unknown message} and \enquote{Invalid token} while the JSON objects are used for more complex results such as arrays of matches.

% Client requests
These responses are to the clients messages.
Currently, there are two messages which are as follows:

{\centering
	\begin{tabular}{l l}
		Update matches & \enquote{\texttt{UPD:[TOKEN]}}\\
		Register & \enquote{\texttt{REG:[TOKEN]:[PHONENUMBER]:[E-MAIL]:[FULL NAME]}}
	\end{tabular}
}

The server checks the token sent through the \gls{astep} API and gets the username in both cases.
Update matches gets the updated set of matches for the user and register adds the user to the user details table in a database on the server.

%Main loop
The communication is contained in a loop that is implemented so that the server is ready for new clients as often as possible.
The loop is as seen in the following enumeration.

\begin{enumerate}
	\item Load server configurations
	\item Open Server side socket
	\item If it is less than 24 hours since last match update from \gls{astep} goto 5
	\item Create a thread to get match updates from \gls{astep}
	\item Wait for a client
	\item Create a thread to handle client request in another thread
	\item goto 3
\end{enumerate}

The 24-hour interval, in line 3, was chosen to minimize the work that the \gls{astep} server has to do while presenting the users with relevant information.
When getting matches from \gls{astep} the system runs the algorithms as described in Section \ref{ssec:algorithm} on routes that were received within the last two week.
The matches are then stored in the server's own database.

%Database
All contact to the database is handled in a class with a single instance ensured with the use of singleton pattern. 
This is because the database needs to know the configurations for the database in question (the username and password), which is loaded from a file.
It would be inefficient if the server had to read this file every time it needs to access the database.

Withing the database is only two tables; userdetails and matches.
The userdetails table keeps extra details about the users of the system such as contact information and an alias they can use apart from their \gls{astep} username (stored as fullname).
The table description can be seen in Table \ref{tab:userdetails}.

\begin{table}[h]
	\centering
	\begin{tabular}{l|l|l}
		Column 		& Type                   & Modifiers\\\hline
		username    & character varying(50)  & primary key\\
		phonenumber & character varying(30)  & not null\\
		email       & character varying(255) &\\
		fullname    & character varying(255) & not null\\
	\end{tabular}
	\caption{The table definition of userdetails in the \gls{rs} system.}
	\label{tab:userdetails}
\end{table}

A table of meta data for matches is stored that contains the information that is required by the system design.
The table description can be seen in Table \ref{tab:matches}.

\begin{table}[h]
	\centering
	\begin{tabular}{l|l|l}
			   Column  &            Type             & Modifiers\\\hline
			 user1     & character varying(50)       & references userdetails(username)\\
			 user2     & character varying(50)       & references userdetails(username)\\
			 starttime & timestamp without time zone & not null\\
			 score     & integer                     &
	\end{tabular}
	\caption{The table definition of matches in the \gls{rs} system.}
	\label{tab:matches}
\end{table}
