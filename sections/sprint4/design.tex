\section{Design}
% metatext
In this chapter the design of the implementation that is to be done in sprint 4 will be described.
The primary focus for the implementation is, as described in analysis, finishing the app and implementing the RideShare server.
A important part is especially implementation of the api calls to the \gls{astep}.

\subsection{Application}
On the application the api call should be implemented.
These calls should be of the format accepted by \gls{astep}, its important to ensure correctness here or the app will not function as intended.
The actual api call connection(A api client) should generalized to a single class so it is easy to add more different calls in the future.
Every api call must be in their own class building the correct URL to be used in a api client.
It is also very important to make the api calls asynchronous. 
If this is not achieved the app will stall at every api call and the UI will have to wait for the api client to do its thing before unfreezing again.
A list of the api calls that must be implemented can be found on table \ref{tab:relevantastepapi}

\subsection{RideShare server}
As it were described in section \ref{sec:s2systemdesign} it were decided that user information not stored on the \gls{astep} server should be stored on the RideShare server.
It were also decided in \ref{sec:s2systemdesign} that it is the RideShare server which must call the matched using the api and then stored the matched so it can be delivered to the app. 
In order to do so a  database is needed on the RideShare server.
The database must contain two tables, one for user data and one for matched routes.
The database must be accessible from the developed server executable.

As there a little functionality on the RideShare Server a simpler implementation of the api calls can be implemented.
Both the api calls and the connection to the app should be implemented in asynchronous as there can be several request going to and from the server simultaneously. 
If this is not done request can be blocked and functionality of both the server and app may decrease.