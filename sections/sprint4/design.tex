\section{Design}
% metatext
The design of the implementation planned for sprint 4 is described in this chapter.
The primary focus for the implementation is finishing the app and implementing the \gls{rs} server.
An important part is the implementation of the API calls to the \gls{astep}, as these are fundamental with regards to logging into the \gls{astep} system and to retrieve the route matches from \gls{astep}.

\subsection{\gls{rs} App}
To make the \gls{rs} app acceptable by fulfilling the 'Must have' requirements, the missing features from Section \ref{sec:req} must be implemented.
A list of matches needs to be implemented so that the user interface reflects the respective users' matches.
To implement the user matches list, the app must utilize the API calls regarding route matches.
These calls must be in the correct \gls{astep} format to be accepted and to work properly.
The actual API call could be generalized to a single class so it can be reused for different API calls and to support potential future expansion.
Every API call must be in their own class for building the correct URL to be used with an API client.

It is important to make the API calls asynchronous for the user experience and to avoid user interface stalls. 
If the functions are made synchronous, the app will wait for a response for the API calls, and hence rendering the user interface unusable.
The stall is affecting the whole app, and it could lead to other issues or failures.
A list of the API calls that must be implemented to make the \gls{rs} app functional can be found in Table \ref{tab:relevantastepapi}.

\subsection{\gls{rs} Server}
As described in Section \ref{sec:s2systemdesign}, it was decided that additional user information not stored in the \gls{astep} server should be stored in a \gls{rs} server.
It was also decided in Section \ref{sec:s2systemdesign} that it is the \gls{rs} server that is responsible for calling the \gls{astep} API to store the calculated matches, thus enabling the \gls{rs} server to provide the route match information for the \gls{rs} app.
In order to store the data received from \gls{astep}, a database is needed on the \gls{rs} server.
The database needs two tables to perform the required operations: one for user data and one for matched routes.

As there are few tasks performed on the \gls{rs} server, a simpler implementation of the API calls can be implemented.
Both the API calls and the connection to the app should be implemented as asynchronous because there can be several requests to and from the server simultaneously. 