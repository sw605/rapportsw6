\section{Requirement Specification}\label{sec:req}
% Metatext
%To provide a clear direction of the solution and to state the success criteria, this section contains the requirements for the application to be developed. 
The requirements are divided into two main categories; functional and nonfunctional requirements and are furthermore sorted in the MoSCoW \cite{moscow} structure.
This assist in solving the highest prioritized requirements first, hence the core purpose of the system can be developed first and the adding additional functionality.


% Requirements
\subsection{Functional requirements}
Functional requirements are requirements directly related to the tasks and operations of the developed application.

\textbf{Must-have requirements}\\
These requirements must be fulfilled to have an acceptable solution.

\textit{Graphical user interface}\\
The application must have a graphical user interface (GUI) as users must be able to operate the application themselves. 
The GUI must be intuitive as users may have different levels of experience with using mobile applications.

\textit{User accounts, including login and registration.}\\
Unique user accounts are required as it serves as an identifier for each user, using the system. 
With user accounts, store personal information, and compare user based on the data. 
It will also provide functionality for displaying user to other users.

\textit{Communication with the \gls{astep} system.}\\
As the project is a part of the bigger system \gls{astep}, there must be a relation to the developed platform. 
The communication would be storing data in the \gls{astep} database and using user management also implemented on the platform.

\textit{User location tracking and storage.}\\
As the application must track users as they move around, location sensors must be polled, usually GPS. 
The application should track users commute.
This data should be stored in the \gls{astep} database.

\textit{Automatically determine regular routes.}\\
Automatically determine regular routes the user take. 
Regular routes should be saved for later comparison. 
Ignoring routes that are not used often.

\textit{Automatic suggest ridesharing partners.}\\
An automatic comparison with other users routes must be computed.
When two regular routes are found similar, the users must receive an indication of the match in the app.


\textbf{Should-have requirements}\\
The requirements in this subsection are important, but not regarded as critical for a functional system.

\textit{Give the user option to specify wherever they have a car or not.}\\
Some users might not have a car, and that should be considered as people whiteout car would need to be treated differently from those who have.

\textit{Enable users to blacklist other users.}\\
Giving they are frequent users of the application, and have a bad experience with either a driver or passenger, they should be able to blacklist them.

\textit{Suggest rides with users who only drives a subset of the way from A to B.}\\
Giving users the option to drive with people who do not have the same source and destination, will increase the pool of which users to suggest, as a user can ``tag along'' part of the ride.


\textbf{Could-have requirements}\\
These are the lowest realistic requirements. 

\textit{Ride reservation or request from, to, time.}\\
Reserve rides with other users. 
This includes both regular commutes and commutes users would do rarely.

\textbf{Would-have requirements}\\
The following requirements are only considered when all other requirements are satisfied but initially regarded as tasks to be solved in future projects.

\textit{Inform users of their environmental and economic savings due to their use of the solution.}\\
Provide detailed information on how much fuel users save, how much money saved based on fuel prices, and how much CO2 that is not released into the environment. 

\subsection{Non-functional requirements}
The nonfunctional requirements for the solution are stated here.

\textbf{Must-have requirements}\\
These are the requirements the solution must fulfill to be acceptable.

\textit{Development cooperation with the other \gls{astep} project groups.}\\
The development must be done in cooperation with the other \gls{astep} groups.

\textit{User privacy}\\
The app must respect user privacy, especially in regards to a user location data and user data in general.
A solution will be developed in collaboration with other \gls{astep} groups.
The solution should consider elements such as database storage, user management, and developers.

\textbf{Should-have requirements}\\
Requirements that are important, but not regarded as highly critical.

\textit{Aesthetics matching other \gls{astep} project applications}\\
The app must share design and guidelines as the other apps developed for \gls{astep}.

No further non-functional requirements are set at this stage of the project.