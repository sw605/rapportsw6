\section{Requirement Specification}
% Metatext
To provide a clear direction of the solution and to state the success criteria, this section contains the requirements for the solution. 
The requirements are divided into two main categories: functional and nonfunctional requirements. 
The requirements are sorted in the MoSCoW structure, enabling the project group to solve the highest prioritized requirements first, hence developing a solution that fulfills the core purpose before adding additional functionality.
\todo{add reasoning and source of MSCW categories + functional/non-functional}


% Requirements
\subsection{Functional requirements}
Functional requirement are requirements directly related to the tasks and operations of the developed application.

\textbf{Must-have requirements}\\
These are the requirements the solution must fulfill to be acceptable.

\textit{Graphical user interface}\\
The application must have a graphical user interface (GUI), as user must be able to operate the application them selves. 
The GUI must be intuitive, as user may have different experience with using mobile applications.

\textit{User accounts, including login and registration.}\\
Unique user accounts is required as it serve as a identifier for each user, using the system. With user accounts, store personal information, and compare user based on the data. 
It will also provide function for displaying user to other users.

\textit{Communication with the aSTEP core.}\\
As the project is a part of the bigger system aSTEP, there mush be a relation to the developed platform. 
The communication would be storing data in the aSTEP database and using user management also implemented on the platform.

\textit{User location tracking and storage.}\\
As the application must track users as they move around, GPS locations are needed.\todo{source} 
The application should track users whenever they are in a vehicle, and store the data in the aSTEP database. \todo{why?}

\textit{Automatically determine regular routes.}\\
Automatically determine regular routes the user take. 
Regular routes should be saved for later comparison. 
Discarding routes that are not used often.

\textit{Automatic ride sharing recommendations.}\\
A automatically comparison with other users routes must be computed. When two regular routes are found similar, each user must be notified of such, whiteout any user input.


\textbf{Should-have requirements}\\
The requirements in this subsection are requirements that are important, but not regarded as highly critical.

\textit{Give the user option to specify wherever they have a car or not.}\\
Some users might not have a car, and that should be considered as people whiteout car would need to be treated differently from those who have.

\textit{Enable user to blacklist other users.}\\
Giving they are frequent users of the application, and have a bad experience with either a driver or passenger, they should be able to blacklist them.

\textit{Suggest rides with user who only drives a subset of the way from A to B.}\\
Giving users the option to drive with people who do not have the same source and destination, will increase the pool of which users to suggest, as user can \"tag along\" part of the ride.


\textbf{Could-have requirements}\\
These are the lowest realistic requirements. 

\textit{Ride reservation or request from, to, time.}\\
Reserve rides with other users. 
This includes both regular commutes and commutes users would do rarely.

\textbf{Would-have requirements}\\
The following requirements are only considered when all other requirements are satisfied, but initially regarded as tasks to be solved in future projects.

\textit{Inform users of their environmental and economical savings due to their use of the solution.}\\
Provide detailed information of how much fuel users save, how much money saved based on fuel prices, and how much CO2 that is not released into the environment. 

\subsection{Non-functional requirements}
The nonfunctional requirements for the solution are stated here.

\textbf{Must-have requirements}\\
These are the requirements the solution must fulfill to be acceptable.

\textit{Development cooperation with the other aSTEP project groups.}\\
The development of the app must be done in cooperation with the other aSTEP project groups.

\textit{User privacy}\\
The application must have cooperation with other groups also working on the aSTEP project. Such as, database storage, user management and other developers.\todo{rewrite this}

\textbf{Should-have requirements}\\
The requirements in this subsection are requirements that are important, but not regarded as highly critical.

\textit{Aesthetics matching other aSTEP project applications}\\
The application must be of the same design and guidelines as the other applications developed for aSTEP.

\textbf{Could-have requirements}\\
These are the lowest realistic requirements. 

\textit{Placeholder}\\
\todo{find requirement}

\textbf{Would-have requirements}\\
The following requirements are only considered when all other requirements are satisfied, but initially regarded as tasks to be solved in future projects.
\todo{find requirement}

