\section{Requirement Specification}
% Metatext
To provide a clear direction of the solution and to state the success criteria, this section contains the requirements for the solution. The requirements are divided into two main categories: functional and nonfunctional requirements. The requirements are sorted in the MoSCoW structure, enabling the project group to solve the highest prioritized requirements first, hence developing a solution that at least fulfills the core purpose.


% original requirements
\iffalse
\begin{itemize}
	\item Opret brugere, og bestem disposition af bil
	\item Registrere all selv kørte strækninger
	\item Sammenlign med andre med (ca) samme start og sluts position
	\item Udregn strækning mellem sammenligningerne. 
	\item Automatisk forslag om ridesharing mellem pendlere, som tit rejser fra og til samme område på samme tid, hvor tid og penge kunne spares.
	\item Reservation af køre [fra, til, klokkeslet] (normal carpool)
\end{itemize}
\fi


% Requirements
\subsection{Functional requirements}
The functional requirements for the solution are stated here.

\textbf{Must-have requirements}\\
These are the requirements the solution must fulfill to be acceptable.
\begin{itemize}
	\item Graphical user interface
	\item User accounts, including login and registration
	\item Communication with the aSTEP core
	\item User location tracking and storage
	\item Automatic ride sharing recommendations
\end{itemize}

\textbf{Should-have requirements}\\
The requirements in this subsection are requirements that are important, but not regarded as highly critical.
\begin{itemize}
	\item something
\end{itemize}

\textbf{Could-have requirements}\\
These are could-have requirements
\begin{itemize}
	\item Ride reservation or request from, to, time.
\end{itemize}

\textbf{Would-have requirements}\\
The following requirements are only attacked when all other requirements are satisfied, but initially regarded as tasks to be solved in future projects.
\begin{itemize}
	\item Inform users of their environmental and economical savings due to their use of the solution
\end{itemize}

\subsection{Non-functional requirements}
The nonfunctional requirements for the solution are stated here.

\textbf{Must-have requirements}\\
These are the requirements the solution must fulfill to be acceptable.
\begin{itemize}
	\item Development cooperation with the other aSTEP project groups.
	\item User privacy
\end{itemize}

\textbf{Should-have requirements}\\
The requirements in this subsection are requirements that are important, but not regarded as highly critical.
\begin{itemize}
	\item Aesthetics matching other aSTEP project applications
\end{itemize}

\textbf{Could-have requirements}\\
These are could-have requirements
\begin{itemize}
	\item Placeholder
\end{itemize}

\textbf{Would-have requirements}\\
The following requirements are only attacked when all other requirements are satisfied, but initially regarded as tasks to be solved in future projects.
\begin{itemize}
	\item Inform users of their environmental and economical savings due to their use of the solution
\end{itemize}
