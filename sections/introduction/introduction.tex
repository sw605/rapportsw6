\chapter{Introduction}\label{ch:introduction}
%Structure
% Introduction
% Problem domain & environment
%	- Semester project, cooperation with other groups
% Intention of the solution
%	- Scope
%	- Backend focus, not GUI and aesthetics
%	- Advantages of such a solution
% Transition to problem statement


% Introduction - awesome first sentence
Commuters are causing traffic jams in dense areas where people are traveling similar routes, this project embarks the task of informing the individual commuters of other commuters traveling a similar path, so that they can share vehicle.
A such task also seeks to reduce the commuters impact on the environment.

\section{Project environment}
% Problem domain and environment
This project is part of a collaboration project, \gls{astep}, of the SW6F16 semester.
The semester goal is to develop a location-based service with accompanying applications that utilizes the service. 

% Semester project environment + sprint description
The \gls{astep} project is developed in 10 teams in the span of 4 sprints.
The sprints are synchronization points of the project groups, and this report will contain documentation of each sprint chronologically.
The first sprint is intended for analysis and collaboration development, the second and third are development sprints, and the fourth is intended for testing and finalizing the development.
The 10 project groups are assigned to their respective tasks, with 7 project groups developing the core \gls{astep} service, while the remaining three groups are developing applications utilizing the system.

\section{Problem domain}
% Solution scope and definition
\gls{astep}'s planned functionality gave the idea of a large set of location data which a system could use to identify and help drivers and passengers with arranging ridesharing.
The solution's goal is to provide ridesharing suggestions to the users of the solution.
When referring to the term ridesharing in this report, it is referred to as the action of private persons sharing a car for the whole or a part of a route. 
This definition is covered by the terms carpool and ad-hoc ridesharing by \citet{doi:10.1080/01441647.2011.621557}.  

% Intention of the solution
The intention of this project is to produce an android app that utilizing the \gls{astep} system for historical location data which should ease the arranging carpool-style ridesharing.
The scope of such a project can be large and therefore we choose early that the main focus will be to design and implement the basic system which then later can be expanded on.
The purpose of the application will be to propose appropriate candidates for ridesharing.
This will revolve around develop and implementing an algorithm which asses if two users of the system should be paired for ridesharing.
Furthermore, other parts of the application such as graphical user interface is not prioritized, but is developed to be functional enough for test and practical purposes.

% Solution contributions/advantages
The suggested app could be beneficial in several ways.
There could be environmental and economical savings, as the number of polluting cars driven is reduced as described in \cite{doi:10.1080/01441647.2011.621557}.
The app would ideally decrease problems such as traffic jam and thereby also being time-saving, and causing less frustration for the drivers.

The potential benefits are both social, environmental and economical, as the app allows people meet each other and thus reduce the use of cars, gasoline and the load on the road system.
The future aspect of this solution seems bright since it easily should modified for future transportation needs, i.e. autonomous vehicles.

% Transition to problem statement
To formally define the problem, to initiate research and development, a concrete problem formulation is made.