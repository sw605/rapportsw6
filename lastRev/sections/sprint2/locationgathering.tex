\subsection{Location Gathering}
% metatext
To get the user location in the background service\todo{describe BGS before this}, the Google Play Services is utilized.
The Google Play Services provides necessary location data, due to the integration on the platform.

% permissions
First, to retrieve location information, the \texttt{manifest.xml} file must be edited as to acquire permissions to the course and fine location.
When the manifest is set up correctly, the actual location retrieving can be performed.
This is done by adding the following lines to the manifest:
\begin{lstlisting}[language=XML]
<uses-permission android:name="android.permission.ACCESS_COARSE_LOCATION"/>
<uses-permission android:name="android.permission.ACCESS_FINE_LOCATION"/>
\end{lstlisting}

% gathering overview
The location is retrieved by utilizing the \texttt{GoogleApiClient}, then converted to an \gls{astep} location format.\todo{ensure this is documented}
This prepares the location data to be sent to the aSTEP system.


% googleApiClient
\textbf{Google API Client}\\
The \texttt{GoogleApiClient} is built and connected to when the \texttt{JobService} is created, hence performed in the \texttt{onCreate()} method, as the \texttt{JobService} thread only exists for the execution time of the contents.
The \texttt{GoogleApiClient} is disconnected in the method \texttt{onDestroy()} when the thread is terminated.

\iffalse
\begin{lstlisting}[caption={onCreate()},label={lst:oncreate},language=Java]
@Override
public void onCreate() {
	// Set the current Context to this
	context = this;
	// Builds the Google API Client to enable location
	buildGoogleApiClient();
	// Connects the Google API Client. It broadcasts to onConnect() when connected.
	connectApi();
}
\end{lstlisting}
\fi



% onConnected()
The \texttt{onConnected(Bundle bundle)} method is called when the Google API Client is ready.
When the client is ready, the \texttt{getCurrentGLocation()}, which can be seen in \ref{lst:onconnected}, is called, and the returned Android Location is stored in \texttt{currentGLocation}. 
The current Google location can then be stored in an aSTEP object, that later can be transmitted to the aSTEP system.

\begin{lstlisting}[caption={onConnected()},label={lst:onconnected},language=Java]
@Override
public void onConnected(Bundle bundle) {
	currentGLocation = getCurrentGLocation();

	// convert location to astep if available
	if (currentGLocation != null) {
		ASTEPLocation currentAstepLocation = convertToAstepLocation(currentGLocation);
	}
}
\end{lstlisting}


% getCurrentLocation()
\textbf{Get Location}\\
%To get the location of the device, the \texttt{getCurrentGLocation()} is called when the \texttt{onConnected(Bundle bundle)} receives the broadcast signal\todo{what?}.
The \texttt{getCurrentGLocation()} returns the last known location, according to the \texttt{googleApiClient}, in the Android Location format.
Before the location is gathered, because the target API is 23, the method needs to confirm the permissions required to acquire location data.
This is performed on line 6 and 7 seen on \ref{lst:getCurrentLocation}.
The device location is gathered through the \texttt{FusedLocationApi.getLastLocation()} method, using the \texttt{googleApiClient} as argument.

\begin{lstlisting}[caption={getCurrentLocation},label={lst:getCurrentLocation},language=Java]
private Location getCurrentGLocation(){
	Location tempGLocation = null;
	
	// Support Android M type permission handling
	try {
		if (ActivityCompat.checkSelfPermission(context, Manifest.permission.ACCESS_FINE_LOCATION) != PackageManager.PERMISSION_GRANTED &&
		ActivityCompat.checkSelfPermission(context, Manifest.permission.ACCESS_COARSE_LOCATION) != PackageManager.PERMISSION_GRANTED) {
			Log.d("BGS-LOC", "pass permission test");
			tempGLocation = LocationServices.FusedLocationApi.getLastLocation(googleApiClient);
		}
		
		// Get the current googleApiClient/LocationServices location
		tempGLocation = LocationServices.FusedLocationApi.getLastLocation(googleApiClient);
		
	}catch (Exception ex){
	}

return tempGLocation;
}
\end{lstlisting}

% aSTEP location
\textbf{Convert to aSTEP location}\\
The aSTEP location format contains the essential location information: latitude, longitude, accuracy, and a timestamp.
The \ref{lst:asteplocation} takes an Android location format data, creates an aSTEP location instance based on the argument location, and returns the aSTEP location.

\begin{lstlisting}[caption={ASTEPLocation()},label={lst:asteplocation},language=Java]
private ASTEPLocation convertToAstepLocation(Location gLocation) {
	Log.d("BGS-LOC", "convertToAstepLocation");
	ASTEPLocation aLocation = new ASTEPLocation(gLocation.getLatitude(),
	gLocation.getLongitude(),
	gLocation.getAccuracy(),
	gLocation.getTime());
	return aLocation;
}
\end{lstlisting}


