\subsection{JobScheduler}
The \texttt{JobScheduler} was chosen as the scheduler for the background service in Section \ref{ssec:periodictasks}.
The \texttt{JobScheduler} works by scheduling Intents, that are operations to be performed. %what you would normally use to call different activities called an Intent.
An \texttt{Intent} that is not yet executed, is called a \texttt{PendingIntent}.
When a \texttt{PendingIntent} is finally executed, variables in the systems may have been changed, hence the context might not be the correct, therefore a \texttt{PendingIntent} also requires a reference to the context when it was created.
This is performed to ensure that permissions given to the app are still available when the \texttt{job} is executed.
A \texttt{job} is a \texttt{PendingIntent} as described in Section \ref{ssec:periodictasks} and a set of requirements for when the intent should be executed.

When a job is handed to the \texttt{JobScheduler}, information about the nature of the scheduling is also provided.
This includes requirements that needs to be fulfilled before the job can run, how often the job should be executed, and how precise the timing of the job should be.

%The job itself can tell the scheduler how successful the job was, and if the job was unsuccessful and should be rescheduled. \todo{relevant?}

\begin{lstlisting}[language=Java, label=jobScheduler, caption=The implementation of jobScheduler.]
JobScheduler jobScheduler = (JobScheduler) getSystemService(JOB_SCHEDULER_SERVICE);
JobInfo.Builder builder = new JobInfo.Builder(1, new ComponentName(getPackageName(), RideShareService.class.getName()));
builder.setPeriodic(10000);
int response  = jobScheduler.schedule(builder.build());	
\end{lstlisting}

Listing \ref{jobScheduler} presents the implementation of \texttt{jobScheduler} in the \texttt{MainActivity} of the app.

In the app, the job is implemented as an extension to the \texttt{JobService} class which provides the required interface for the \texttt{JobScheduler} called RideShareService.
The class reroutes the call by the \texttt{JobScheduler} to a \texttt{Handler} class.
This \texttt{Handler} class, as described by the official documentation \cite{handler}, allows the service to send a runnable object to the threads message queue.
%This is then executed in the UI thread for debugging.
%The contents of this handler will be further explained later. \todo{Check that this line is correct.}

The implementation makes the location gathering be performed as a background service every 10 seconds.
The interval is temporary, and will serve to test the implementation.


